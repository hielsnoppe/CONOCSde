\documentclass[german]{scrbook}
\usepackage[ngerman]{babel}
\usepackage[utf8]{inputenc}
\usepackage{algo}
\usepackage{amssymb}
\usepackage{amsmath}
\usepackage{marvosym}
%\usepackage[T1]{fontenc}
\usepackage[pdftex, bookmarks=true, bookmarksnumbered=true]{hyperref}

% convenience
\newcommand{\Nat}{\mathbb{N}}
\newcommand{\Zahl}{\mathbb{Z}}
\newcommand{\Rat}{\mathbb{Q}}
\newcommand{\Real}{\mathbb{R}}
\newcommand{\LandauO}{\mathcal{O}}
\newcommand{\Landauo}{\mathcal{o}}
\newcommand{\LandauOmega}{\Omega}
\newcommand{\Landauomega}{\omega}
\newcommand{\LandauTheta}{\Theta}
\newcommand{\floor}[1]{\left\lfloor{#1}\right\rfloor}
\newcommand{\ceil}[1]{\left\lceil{#1}\right\rceil}
\newcommand{\abs}[1]{\left|#1\right|}

\title{Studium der Informatik}
\subtitle{Ein Reiseführer}
\author{Niels Hoppe}

% \part{}, \chapter{}, \section{}, \subsection{}, \subsubsection{}, \subsubsubsection{}, \paragraph{}, \subparagraph{}
\begin{document}

\maketitle
\tableofcontents

\part{Algorithmen und Programmieren}
\chapter{Funktionale Programmierung}
\chapter{Objektorientierte Programmierung}
\chapter{Datenstrukturen und Datenabstraktion}
%
%
%
%
\section{Motivation}

Definition: Algorithmus\\
Ein Algorithmus ist eine Rechenvorschrift, die eine Eingabe in eine Ausgabe überführt, und:
\begin{enumerate}
	\item endlich beschrieben,
	\item effektiv,
	\item allgemein,
	\item (deterministisch)
\end{enumerate}
ist.
Ein Algorithmus ist gut, wenn er:
\begin{enumerate}
	\item immer korrekt terminiert.
	\item richtige Antworten gibt.
	\item schnell,
	\item speichereffizient,
	\item gut lesbar
\end{enumerate}
ist.

Bewertung nach Laufzeit und Speicherbedarf, experimentell oder theoretisch\\
Analyse-Verfahren nach den Datenstrukturen und Algorithmen\\
Zum Schluss Anwendungen

\subsection{Bewertung und Analyse von Algorithmen}

Analyse auf Basis von Pseudo-Code, eine Zeile PC entspricht beliebig aber konstant vielen Schritten auf RAM.
%
%
\subsubsection{Registermaschinen und O-Notation}
Empirische Analyse vs. theoretische Analyse\\
Definition: Registermaschine bzw. Random Access Machine (RAM)\\
Definitionen: Laufzeit und Speicherbedarf bei einer RAM
%
%
\subsubsection{Amortisierte Analyse}
Wir betrachten eine Folge von Operationen eines Algorithmus im worst-case.
Wenn wir zeigen können, dass jede Folge von $n$ Operationen $f(m)$ Zeit benötigt,
dann sagen wir, die amortisierten Kosten pro Operation sind $f(m)/n$.
%
%
\subsubsection{Analyse randomisierter Algorithmen}


\section{Programmiertechniken}
\label{sec:Programmiertechniken}
%
%
\subsection{Rekursion}
\label{subsec:Rekursion}

%
%
\subsection{Dynamische Programmierung}
\label{subsec:DynamischeProgrammierung}
Dynamische Programmierung ist eine Technik um die Effizienz rekursiver Alg. zu verbessern.
Dabei werden Zwischenschritte gespeichert, die sonst bei erneuten Rekursionsaufrufen erneut berechnet würden.
Dazu wird eine Tabelle verwendet, in die ...

%
%
\subsection{Gierige Algorithmen}
\label{subsec:GierigeAlgorithmen}

Gierige (engl. \textit{greedy}) Algorithmen treffen lokal optimierte Entscheidungen in der Erwartung so auch ein globales Optimum zu erreichen.

Ein Beispiel dafür ist das Münzwechseln

Weitere gierige Algorithmen, die im weiteren Verlauf vorgestellt werden, sind zum Beispiel
der Kompressionsalgorithmus von Huffman (\ref{subsubsec:HuffmanKodierung}),
die Tiefensuche (?) in Graphen (\ref{subsec:Tiefensuche}) sowie
die Graphen-Algorithmen von Prim (\ref{subsubsec:PrimJarnikDijkstra}),
Kruskal (\ref{subsubsec:Kruskal}) und Dijkstra (\ref{subsubsec:Dijkstra}).


%
%
\subsection{Backtracking}
\label{subsec:Backtracking}

%
%
\subsection{Hashing}
\label{subsec:Hashing}
Zur Implementierung in Java:
http://www.pakzilla.com/2009/08/24/hashmap-vs-hashtable-vs-hashset/

\subsubsection{Konfliktbehandlung}

\paragraph{Verkettung}
\paragraph{Lineares Sondieren}
\paragraph{Kuckuck} Bei diesem Verfahren werden zwei unterschiedliche Hashfunktionen $h_1$ und $h_2$ verwendet.
Sobald eine Kollision entsteht, nimmt das neue Element den Platz des bereits vorhandenen ein und dieses wird mit der anderen Hashfunktion erneut gehasht.

!!! Mit welcher genau? Immer abwechselnd, oder für jedes Element verwendete Funktion merken? !!!

\subsubsection{Universelles Hashing}
Menge von Hashfunktionen $H$, Menge von Schlüsseln $K$.
$h \in H$ ist universell, wenn $Pr(h(k_1) = h(k_2)) \leq \frac{|H|}{|N|}$.
$H$ ist universell, wenn alle $h \in H$ universell (oder so).
$H$ ist universell, wenn $\forall h \in H: \forall k_1, k_2 \in K, k_1 \neq k_2: Pr(...)$


\section{Einfache Datentypen}
\label{sec:EinfacheDatentypen}
%
%
\subsection{Dynamische Arrays}
\label{subsec:DynamischeArrays}
%
%
\subsection{Listen}
\label{subsec:Listen}
Einfach oder doppelt verkettet.
%
%
\subsection{Stacks (Stapel)}
\label{subsec:Stacks}

\subsubsection{Operationen}

\paragraph{push} Fügt ein Element oben auf dem Stapel hinzu.
\paragraph{pop} Gibt das oberste Element des Stapels zurück und entfernt es.
\paragraph{top} Gibt das oberste Element des Stapels zurück.
\paragraph{size} Gibt die Anzahl der Elemente des Stapels zurück.
\paragraph{isEmpty} Gibt \textit{wahr} zurück, wenn die Warteschlange leer ist, sonst \textit{falsch}.

\subsubsection{Implementierung}
Implementiert als verkettete Liste oder (dynamisches) Array
%
%
\subsection{Queues (Warteschlangen)}

\subsubsection{Operationen}
\paragraph{enqueue} Fügt ein Element an das Ende der Warteschlange an.
\paragraph{dequeue} Gibt das vorderste Element der Warteschlange zurück und entfernt es.
\paragraph{front} Gibt das vorderste Element der Warteschlange zurück.
\paragraph{size} Gibt die Anzahl der Elemente der Warteschlange zurück.
\paragraph{isEmpty} Gibt \textit{wahr} zurück, wenn die Warteschlange leer ist, sonst \textit{falsch}.

\subsubsection{Implementierung}
Implementiert als verkettete Liste oder (dynamisches) Array\\

\subsection{Priorityqueues (Pritoritätswarteschlangen)}
\label{subsec:Priorityqueues}

\subsubsection{Operationen}
\paragraph{insert / enqueue} Fügt ein Element an das Ende der Warteschlange an.
\paragraph{deleteMin / dequeue} Gibt das vorderste Element der Warteschlange zurück und entfernt es.
\paragraph{findMin / front} Gibt das vorderste Element der Warteschlange zurück.
\paragraph{size} Gibt die Anzahl der Elemente der Warteschlange zurück.
\paragraph{isEmpty} Gibt \textit{wahr} zurück, wenn die Warteschlange leer ist, sonst \textit{falsch}.

\subsubsection{Implementierung}
Eine Prioritätswarteschlange kann ähnlich einer normalen Warteschlange implementiert werden, wenn die Funktion zum Einfügen die Prioritäten berücksichtigt.
Effizienter lässt sich eine Prioritätswarteschlange jedoch durch Heaps (\ref{subsec:Heaps}) implementieren, die später im Abschnitt über Bäume (\ref{sec:Baeume}) erklärt werden.


\section{Wörterbücher}

Der abstrakte Datentyp Wörterbuch (engl. \textit{dictionary}) ermöglicht es, Tupel zu speichern,
die aus Schlüsseln $k$ (engl. \textit{keys}) aus einer Schlüsselmenge $K$ bestehen,
sowie aus Werten $v$ (engl. \textit{values}) aus einer Wertemenge $V$,
auf die die Schlüssel abgebildet werden.

Je nachdem, ob die Schlüsselmenge geordnet ist oder nicht,
spricht man von einem ungeordneten oder einem geordneten Wörterbuch.

%
%
\subsection{ungeordnetes Wörterbuch}

\subsubsection{Operationen}
\paragraph{put $k$ $v$} Fügt einen Wert $v$ mit $k$ als Schlüssel in das Wörterbuch ein.
\paragraph{get $k$} Gibt den Wert zu einem Schlüssel $k$ aus dem Wörterbuch zurück.
\paragraph{remove $k$} Entfernt den Schlüssel $k$ und den dazugehörigen Wert aus dem Wörterbuch.

\subsubsection{Implementierung}

Implementierung durch Hashing (\ref{subsec:Hashing}) mittels Arrays...

%
%
\subsection{geordnetes Wörterbuch}

Bei einem geordneten Wörterbuch geht man von einer total geordnete Schlüsselmenge $K$ aus.
Das heißt, für alle $k_1, k_2, k_1 \neq k_2 \in K$ gilt entweder $k_1 < k_2$ oder $k_2 < k_1$.

\subsubsection{Operationen}
Durch diese Eigenschaft können ergeben sich zusätzlich zu den Operationen eines ungeordneten Wörterbuches folgende Operationen:

\paragraph{min} Gibt den kleinsten Schlüssel des Wörterbuches zurück.
\paragraph{max} Gibt den größten Schlüssel des Wörterbuches zurück.
\paragraph{succ $k$} Gibt den nächstgrößeren Schlüssel als $k$ (Nachfolger, engl. \textit{successor}) aus dem Wörterbuch zurück.
\paragraph{pred $k$} Gibt den nächstkleineren Schlüssel als $k$ (Vorgänger, engl. \textit{predecessor}) aus dem Wörterbuch zurück.

\subsubsection{Implementierung}

Eine einfache Möglichkeit zur Implementierung ist ähnlich wie bei einem ungeordneten Wörterbuch vorzugehen und jeweils Suchoperationen durchzuführen.
Es ist aber leicht zu erkennen, dass dies bei größeren Datenmengen sehr ineffizient ist.

Viele effizientere Möglichkeiten, geordnete Wörterbücher zu implementieren,
lernen wir im folgenden Abschnitt über Bäume (\ref{sec:Baeume}) kennen.


\section{Bäume}
\label{sec:Baeume}

Zu jeder der Datenstrukturen werden im Folgenden alle Operationen eines geordneten Wörterbuches für Elemente $e$ definiert,
die in einer konkreten Implementierung eines Wörterbuches die Schlüssel darstellen.
Zusätzlich müssen dann Verweise auf die jeweils zu einem Schlüssel gehörigen Werte gespeichert werden.
Wie das im Einzelnen realisiert werden kann, ist an den jeweiligen Stellen erläutert.

\begin{center}
\begin{tabular}{|p{.4\linewidth}|p{.5\linewidth}|}
\hline
Datenstruktur		& Laufzeiten	\\\hline\hline
\paragraph{Binäre Suchbäume}
&
\\\hline
\paragraph{AVL Bäume}
&
$$\mathcal{O}(\log n)$$
\\\hline
\paragraph{Rot-Schwarz-Bäume}
&
\\\hline
\paragraph{$(a, b)$-Bäume}
&
\\\hline
\paragraph{Heaps}
&
$$\mathcal{O}(\log n)$$
\\\hline
\paragraph{Tries}
&
\\\hline
\paragraph{Skip-Listen}
&
\\\hline
\end{tabular}
\end{center}

\subsection{Binäre Suchbäume}

Operationen jeweils einmal rekursiv und einmal iterativ
\paragraph{Einfügen}
\paragraph{Suchen}
\paragraph{Löschen}

\subsection{AVL-Bäume}
AVL-Bäume sind binäre Suchbäume, die einem weiteren Kriterium genügen.
Sie vermeiden es nämlich, dass der Baum "`entartet"', also besonders stark ungleichmäßig wächst.
Dadurch verbessert sich die Laufzeit der Operationen auf ihnen.
Die Operationen \textit{Einfügen}, \textit{Suchen} und \textit{Löschen} werden ebenso durchgeführt,
wie in einfachen binären Suchbäumen.
Um jedoch eine Entartung zu vermeiden,
wird ein entstehendes Ungleichgewicht nach dem \textit{Einfügen} oder \textit{Löschen} durch Rotationen ausgegleichen.
Dabei bleibt die binäre-Suchbaum-Eigenschaft erhalten.

\paragraph{Rotation, Doppel-Rotation}

\subsection{Rot-Schwarz-Bäume}
Ein anderes Konzept um die Entartung von binären Suchbäumen zu vermeiden, sind Rot-Schwarz-Bäume.
In ihnen werden jedem Knoten eine Farbe, rot oder schwarz,
zugewiesen und darüber erweiterte Anforderungen an die Struktur des Baumes gestellt:
\begin{itemize}
	\item Die Wurzel des Baumes ist immer schwarz.
	\item Die $\bot$-Knoten sind immer schwarz.
	\item Die Nachfolger roter Knoten sind schwarz.
	\item Auf dem Pfad von der Wurzel zu jedem $\bot$-Knoten liegen immer gleich viele schwarze Knoten.
\end{itemize}

Java-Applet unter http://fbim.fh-regensburg.de/~saj39122/gikasch/start.html

\paragraph{Suchen} Die Suche in einem Rot-Schwarz-Baum verläuft wie in jedem binären Suchbaum.
Ihre asymptotische Laufzeit liegt in $\mathcal{O}(...)$.
\paragraph{Einfügen}
\paragraph{Löschen}

\subsection{$(a, b)$-Bäume}
Anders als die vorigen Baumstrukturen sind $(a, b)$-Bäume keine binäre Stuktur und wachsen nicht von der Wurzel in Richtung ihrer Blätter, sondern umgekehrt.
Anstelle von zwei möglichen Kindern haben die Knoten eines $(a, b)$-Baumes zwischen $a$ und $b$ (jeweils einschließlich, $a, b \in \mathbb{N}$) viele Kinder.
Eine Ausnahme ist die Wurzel, die 2 bis $b$ viele Kinder hat.
Weiterhin enthalten die Knoten selbst zwischen $a-1$ und $b-1$ (s.o.) viele Schlüssel.
Die Suchbaumeigenschaft findet sich darin wieder,
dass die Schlüssel in sortierter Reichenfolge gespeichert werden und die Verweise auf die Kinder zwischen den einzelnen Schlüsseln angelegt sind.
Alle Schlüssel eines Kindes sind also größer als der Schlüssel links und kleiner als der Schlüssel rechts von dem Verweis auf das Kind.

\paragraph{Suchen} Die Suche in einem $(a, b)$-Baum hat eine asymptotische Laufzeit in $\mathcal{O}(\log n)$.

\paragraph{Einfügen} Neue Werte werden in einem $(a, b)$-Baum immer in einem Blatt eingefügt.
Wenn dieses durch den neuen Schlüssel "`überläuft"',
so wird es geteilt und ein Schlüssel daraus in einen übergeordneten Knoten verschoben.
Siehe dazu die Operation \textit{Teilen}.

Das Einfügen hat eine asymptotische Laufzeit in $\mathcal{O}(...)$.

\paragraph{Löschen} Gelöscht werden Werte immer aus den Blättern.
Um einen Wert aus einem inneren Knoten zu löschen,
wird dieser zuerst mit seinem symmetrischen Vorgänger bzw. Nachfolger ersetzt.
Dieser ist immer ein Blatt, sodass er problemlos seine ursprüngliche Position verlassen kann.
In beiden Fällen ist es jedoch möglich, dass anschließend das Blatt, aus dem gelöscht wurde, zu wenige Werte enthält.
Man sagt auch, er "`läuft unter"' analog dazu, wenn ein Knoten überläuft.
Um die Struktur des Baumes dann wieder herzustellen, gibt es unterschiedliche Balancierungsoperationen:

\subparagraph{Auffüllen} (Auch Leihen oder Rotation) Dabei wird ein Wert aus dem Elternknoten in den unterlaufenen Knoten abgegeben.
Um dies wieder auszugleichen wird versucht,
einen Wert von einem benachbarten Geschwisterknoten des unterlaufenen Knotens in den Elternknoten zu verschieben.
Wenn dieser dann aber selbst nicht mehr ausreichend Werte enthält,
wird durch einen Knoten aus dem nächsthöheren Elternknoten ausgegelichen.
Dieser muss dann wiederum durch die gleichen Operationen ausbalanciert werden.

\paragraph{Teilen} Wenn ein Knoten überläuft, so muss er geteilt werden.
Dazu wird der mittlere Wert aus dem übergelaufenen Knoten in dessen Elternknoten eingefügt und alle Werte links und rechts davon in jeweils einen neuen Knoten gespeichert,
der links bzw. rechts von dem aufgestiegenen Wert in dem Elternknoten steht.

\subsection{Heaps}
\label{subsec:Heaps}

Unterscheidung Min-Heap und Max-Heap.\\
Heapbedingung: Alle Kinder eines Knotens größer (Min-Heap) bzw. kleiner (Max-Heap) als Eltern-Knoten.\\
Implementierung als (binärer) Baum verlinkt, oder mit (dynamischem) Array.\\
Als Sonderfall Treaps, die bezüglich zweier Werte einmal Heap- und einmal Binäre-Suchbaum-Eigenschaft haben.

\subsection{Tries}
\label{subsec:Tries}

Tries sind eine Möglichkeit, um Zeichenketten (engl. \textit{strings}) effizient zu verwalten.
Man kann die gespeicherten Zeichenketten auch als Schlüssel für andere Daten verwenden und so ein geordnetes Wörterbuch implementieren.

Die Struktur von Tries sind Bäume, bei denen einzelne Zeichen der Zeichenketten in den Kanten zwischen den Knoten gespeichert werden.
Die Knoten selbst enthalten keine Werte.
Man kann aber beispielsweise die Blätter dazu nutzen Verweise zu speichern, um ein Wörterbuch zu implementieren.
Weiterhin wird zusätzlich zu den Zeichenketten als letztes Zeichen jeweils ein Symbol eingefügt, das das Wortende markiert.
Hier soll dafür das \EURdig-Zeichen dienen.

\begin{figure}[h]%[htbp]
%\includegraphics{unkomprimiertertrie.png}
\caption{unkomprimierter Trie}
\label{fig:UnkomprimierterTrie}
\end{figure}

Um besonders die Suche in Tries zu beschleunigen, kann man komprimierte Tries (auch Patricia-Tries, von engl. \textit{Practical Algorithm to Retrieve Information Coded in Alphanumeric}) einsetzen.
Dabei werden mehrere Kanten, die auf einem Pfad liegen, von dem keine Verzweigungen ab gehen, zu einer zusammengefasst.

\begin{figure}[h]%[htbp]
%\includegraphics{komprimiertertrie.png}
\caption{komprimierter Trie}
\label{fig:KomprimierterTrie}
\end{figure}

Tries können auch als Suffixbaum zum Suchen in Zeichenketten verwendet werden.
Mehr dazu findet sich im Abschnitt über die Arbeit mit Zeichenketten (\ref{subsec:ArbeitenMitZeichenketten}).

\subsubsection{Operationen}

\paragraph{put $e$}
\paragraph{get $e$}
\paragraph{remove $e$}
\paragraph{min}
\paragraph{max}
\paragraph{succ $e$}
\paragraph{pred $e$}

%\subsection{Rank-balanced Trees}
%http://www.cs.princeton.edu/~sssix/papers/rb-trees.pdf

\subsection{Skip-Listen}
\label{subsec:SkipListen}

Skip-Listen sind eine sogenannte "`randomisierte Datenstruktur"'.
Das bedeutet, dass ihre Struktur zwar grundsätzlich festgelegt ist,
jedoch durch die gezielte Verwendung von zufälligen Entscheidungen das genaue Aussehen nicht eindeutig ist.
Auch wenn also mehrfach die gleichen Daten in der gleichen Reihenfolge gespeichert werden,
kann die Datenstruktur jeweils unterschiedlich aussehen.

Die Grundlage für Skip-Listen bilden mehrere einfach verkettete Listen,
die als Ebenen übereinander liegend betrachtet werden.
Auf der untersten Ebene werden Elemente wie in einer normalen sortierten verketteten Liste verwaltet.

Jedes Element kann aber auch in einer zufälligen Anzahl von Ebenen darüber redundant gespeichert sein.
Neben einem Verweis auf seinen Nachfolger innerhalb der Ebene hat es dann auch einen Verweis auf seinen Repräsentanten in der darunterliegenden Ebene.

\begin{figure}[h]%[htbp]
%\includegraphics{skiplist.png}
\caption{Skip-Liste}
\label{fig:SkipListe}
\end{figure}

Dabei verhält es sich so, dass je höher eine Ebene ist,
desto geringer die Wahrscheinlichkeit für ein bestimmtes Element ist, in dieser Ebene vorhanden zu sein.
Dadurch enthalten die höheren Ebenen weniger Elemente als die niedrigeren und können mit weniger Schritten durchlaufen werden.
Diese Eigenschaft ermöglicht es, effiziente Operationen auf der Datenstruktur zu implementieren.

Zur Implementierung eines Wörterbuches kann man zu jedem Element auf der untersten Ebene einen Verweis auf den dazugehörigen Wert speichern.

\subsubsection{Operationen}

\paragraph{put $e$}
\paragraph{get $e$}
\paragraph{remove $e$}
\paragraph{min}
\paragraph{max}
\paragraph{succ $e$}
\paragraph{pred $e$}

\subsubsection{Mathematische Grundlagen}


\section{Graphen}
Implementierung als Adjazenzliste oder Adjazenzmatrix oder verlinkt.

\begin{center}
\begin{tabular}{|p{.4\linewidth}|p{.5\linewidth}|}
\hline
Algorithmus		& Laufzeit	\\\hline\hline
\paragraph{Breitensuche}
&
\\\hline
\paragraph{Tiefensuche}
&
\\\hline
\paragraph{Topologische Sortierung}
&
$$\mathcal{O}(n) \mbox{\quad average, \quad} \mathcal{O}(n^2) \mbox{\quad worst}$$
\\\hline\hline
\paragraph{Dijkstra}
Knoten mit Entfernung zum Startknoten in Prioritätswarteschlange
&
$$\mathcal{O}(|V| \cdot (T_{enqueue} + T_{dequeue})$$
$$ + |E| \cdot T_{decreaseKey})$$
$$\mathcal{O}(|V| \cdot \log |V| + |E| \cdot \log |V|)$$
\\\hline
\paragraph{A*}
Dijkstra mit Heuristik $h$ (zulässig, monoton/konsistent):
\begin{itemize}
\item $\forall v \in V: h(v) \leq d(v)$
\item $\forall (v, w) \in E: h(v) \leq |(v, w)| + h(w)$
\end{itemize}
&
\\\hline
\paragraph{Bellman-Ford}
iteriert $|V|-1$ mal über alle Kanten
&
$$\mathcal{O}(|V| \cdot |E|)$$
$$\mathcal{O}(|V|^2 \cdot |E|)$$
\\\hline
\paragraph{Floyd-Warshal}
iteriert $|V|$ mal über Adjazenzmatrix
&
$$\mathcal{O}(|V|^3)$$
\\\hline\hline
\paragraph{Prim-Jarnik-Dijkstra}
ZSHK wächst von einem Startknoten aus
&
\\\hline
\paragraph{Kruskal}
Wählt aus den leichtesten Kanten, die ZSHKs verbinden
&
\\\hline
\paragraph{Borouvka}
Iteriert über ZSHKs
&
\\\hline
\end{tabular}
\end{center}

\subsection{Breitensuche}
\label{subsec:Breitensuche}

Rekursiver Algorithmus zur Breitensuche

\begin{algorithm}{BFS}[G]{
	\qinput{}
	\qoutput{}
}
Q \qlet new Queue\\ % nach 210: Queue ersetzen mit PriorityQueue()
s.found \qlet \qtrue\\
s.d \qlet 0\\
s.pred \qlet \qnil\\ % eingefügt nach z(=enz) 190
Q.enqueue(s)\\
\qwhile \qnot Q.isEmpty()\\
	v \qlet Q.dequeue()\\
	\qfor $e$ in v.incidentEdges\\
		$w$ \qlet e.opposite(v)\\
		\qif \qnot w.found\\
			\qthen
				w.found \qlet \qtrue\\
				w.d \qlet v.d+1\\
				w.pred \qlet v\\% enz190*\ \*ab210 replaced  w.d<-v.d+elength;w.pred<-v;*\
				Q.enqueue(v)% nach 210: replaced mit Q.insert(w,w.d)
		\qfi
	\qrof
\qelihw
\end{algorithm}

\subsection{Tiefensuche}
\label{subsec:Tiefensuche}

Rekursiver Algorithmus zur Tiefensuche

\begin{algorithm}{DFS}[v_{start}]{
	\qinput{Startknoten $v_{start}$ in einem Graphen}
	\qoutput{}
}
process(v)\\
$v$.visited \qlet \qtrue\\
\qfor all $w$ adjacent to $v$\\
	\qif \qnot $w$.visited\\
		\qthen DFS($w$) \qfi
\qrof
\end{algorithm}

Iterativer Algorithmus zur Tiefensuche

\begin{algorithm}{DFS}[]{
	\qinput{}
	\qoutput{}
}
s \qlet new Stack\\
s.found \qlet \qtrue\\
s.push((s,s.incidentEdges()))\\
\qwhile \qnot s.isEmpty(), (v, e) \qlet s.top()\\
        \qif e.hasNext()\\
		\qthen w \qlet e.next().opposite(v)\\
		\qelse w \qlet \qnil
	\qfi\\
	\qwhile w != \qnil \qand w.found\\
		\qif w = \qnil\\
			\qthen s.pop()\\
			\qelse
				v.found \qlet \qtrue\\
				s.push(w, w.incidentEdges())
		\qfi
	\qelihw
\qelihw
\end{algorithm}

\subsection{Topologische Sortierung}
\label{subsec:TopologischeSortierung}

Eine topoligische Sortierung ist eine Anordnung von Knoten eines Graphen,
bei der keine Kanten von Knoten zu anderen Knoten verlaufen, die vor ihnen angeordnet sind.
Es sind nur azyklische Graphen topologisch sortierbar.

\subsection{Kürzeste Wege}
\label{subsec:KuerzesteWege}

Bei der Suche nach kürzesten Wegen gibt es einige unterschiedliche Teilprobleme:
\begin{itemize}
\item \textbf{single-pair shortest path}
Der kürzeste Pfad zwischen zwei unterschiedlichen Knoten.
\item \textbf{single-source shortest path (SSSP)}
Der kürzeste Pfad zwischen einem Startknoten und allen anderen Knoten des Graphen.
\item \textbf{single-destination shortest path}
Der kürzeste Pfad zwischen einem Endknoten und allen anderen Knoten des Graphen.
Dies ist eine Umkehrung des SSSP und kann durch eine Umkehrung der Kantenrichtungen erreicht werden.
\item \textbf{all-pairs shortest path (APSP)}
Der kürzeste Pfad, von jedem Knoten im Graph, zu jedem anderen Knoten.
\end{itemize}

Kürzeste Wege in einem Graphen können gut durch einen Baum dargestellt werden,
indem zu jedem Knoten ein Verweis auf seinen Vorgänger gespeichert wird.
Es gibt verschiedene Algorithmen, die einige oder alle dieser Teilprobleme lösen und einen solchen Baum erzeugen:

TODO: Teilpfadoptimalität

\subsubsection{Dijkstra}
\label{subsubsec:Dijkstra}

Der Algorithmus von Dijkstra bietet eine Lösung für das \textit{single-pair shortest path}- und das \textit{SSSP}-Problem.
Er ist ein gieriger Algorithmus (s. \ref{subsec:GierigeAlgorithmen} Gierige Algorithmen) und funktioniert nur korrekt, wenn alle Kantengewichte positiv sind.

Zu Beginn des Algorithmus erhalten alle Knoten eine Bewertung,
die ihren minimalen Abstand vom Startknoten darstellt.
Der Startknoten wird daher mit $0$ bewertet, alle übrigen mit $\infty$.
Anschließend werden die Knoten in einer Prioritätswarteschlange verwaltet.
Im weiteren Verlauf wird jedem Knoten ein Vorgänger zugewiesen werden,
sodass in einer Implementierung auch dafür Speicher zu reservieren ist.

Solange die Warteschlange nun also Knoten enthält,
wird jeweils der am niedrigsten bewertete Knoten entnommen und eine Bewertung all seiner Nachbarn vorgenommen:
Für jeden Nachbarn ist der neue Wert die Summe aus dem Wert des aktuellen Knotens und dem Kantengewicht zwischen den beiden Knoten.
Nur, wenn dieser neue Wert kleiner als der alte ist,
wird der entsprechende Nachbar aktualisiert und erhält als neuen Vorgänger den aktuellen Knoten.

Zum Schluss des Algorithmus ist die Warteschlange leer und alle Knoten sind mit der Länge des kürzesten Pfades vom Startknoten zu ihnen bewertet.
Der jeweils kürzeste Pfad lässt sich durch den gespeicherten Vorgänger eines jeden Knotens rekonstruieren.

Wenn nur eine Lösung für das \textit{single-pair shortest path}-Problem gesucht wird,
kann der Algorithmus bereits dann abgebrochen werden, wenn der aktuelle Knoten dem Ziel-Knoten entspricht.

\begin{algorithm}{Dijkstra}[]{
	\qinput{}
	\qoutput{}
}
\end{algorithm}

\paragraph{Laufzeit}
Die Laufzeit des Algorithmus wird maßgeblich durch die Laufzeiten der Operationen auf der verwendeten Datenstruktur bestimmt.
Allgemein lässt sie sich als in
$$\mathcal{O}(|V| \cdot (T_{enqueue} + T_{dequeue}) + |E| \cdot T_{decreaseKey})$$
liegend abschätzen.
Legt man der Prioritätswarteschlange eine Implementierung durch einen binären Heap zu Grunde,
ergibt sich eine Laufzeit in $\mathcal{O}(|V| \cdot \log |V| + |E| \cdot \log |V|)$.

\paragraph{Beispiel}
TODO: beispielhafte Ausführung wie in der Klausur anzuwenden

\subsubsection{A*}
\label{subsubsec:AStern}

Der A*-Algorithmus ist dem Algorithmus von Dijkstra sehr ähnlich.
Um die Laufzeit zu verbessern nutzt er jedoch zusätzlich eine Heuristik $h: V \mapsto \mathbb{R}_0^+$,
die für jeden Knoten eine geschätzte Entfernung zum Zielknoten angibt.
Diese Heuristik ist je nach Anwendungsfall unterschiedlich geschickt zu wählen.
Sie hat jedoch zwei Kriterien zu erfüllen:
\begin{enumerate}
\item Sie muss zulässig sein, d.h. die Weglänge darf nie überschätzt werden.
$$\forall v \in V: h(v) \leq d(v)$$
\item Sie muss monoton oder auch konsistent sein,
d.h. sie ist zulässig und sie muss für jeden Nachbarn eines Knotens einen mindestens so großen Abstand schätzen,
wie die Summe aus der Kantenlänge zu diesem Nachbarn und der Schätzung für den Knoten selbst.
$$\forall (v, w) \in E: h(v) \leq |(v, w)| + h(w)$$
\end{enumerate}
In einem Graphen, der eine Zwei- oder Mehr-Dimensionale Karte repräsentiert, könnte das zum Beispiel die Luftlinie, also der euklidische Abstand sein. \nocite{wiki:astar}

\paragraph{Laufzeit}

\paragraph{Beispiel}
TODO: beispielhafte Ausführung wie in der Klausur anzuwenden

\subsubsection{Bellman-Ford}
\label{subsubsec:BellmanFord}

Anders als der Algorithmus von Dijkstra funktioniert dieser Algorithmus auch mit negativen Kantengewichten,
jedoch dürfen keine negativen Kreise vorhanden sein.
Auch er löst das \textit{single-pair shortest path}- und das \textit{SSSP}-Problem durch eine Bewertung der Knoten mit ihrer Distanz zum Startknoten.
Jedoch nutzt er keine Warteschlange, sondern iteriert über die Knoten und Kanten des Graphen.

Zu Beginn werden wiederum alle Knoten mit $\infty$ bewertet und der Startknoten mit $0$.
Anschließend wird $|V| - 1$ Mal über alle Kanten iteriert und dabei jeweils eine erneute Bewertung für den Endknoten der jeweiligen Kante vorgenommen:
Wenn die Summe aus dem Wert des Startknotens der Kante und dem Kantengewicht kleiner ist als der Wert des Endknotens der Kante, so wird dieser mit dieser Summe als neuem Wert aktualisiert.

\begin{algorithm}{BellmanFord}[]{
	\qinput{}
	\qoutput{}
}
\qfor $i = 1$ \qto $|V| - 1$\\
	\qfor $(v_1, v_2) \in E$\\
		\qif distance of $v_1$ $+$ weight of $(v_1, v_2)$ $<$ distance of $v_2$\\
			\qthen distance of $v_2$ \qlet distance of $v_1$ $+$ weight of $e$
		\qfi
	\qrof
\qrof
\end{algorithm}

\paragraph{Laufzeit}

\paragraph{Beispiel}

\subsubsection{Floyd-Warshall}
\label{subsubsec:FloydWarshall}

Der Algorithmus von Floyd und Warshall ist der erste,
der das \textit{APSP}-Problem und damit alle Teilprobleme kürzester Wege löst.
Sein Ansatz beruht auf der Teilpfadoptimalitätseigenschaft

Knoten durchnummerieren, $n$ Tabellen $n \times n$\\
$d_{i,j}^0 = c(i, j)$\\
$c(i, j) = \infty$ wenn $i$ und $j$ nicht verbunden, $c(i, i) = 0$.\\
$d_{i,j}^k = \min (d_{i,j}^{k-1}, d_{i,k}^{k-1} + d_{k,j}^{k-1})$\\
Rekursiv. Mit dynamischer Programmierung verbessern:

\begin{algorithm}{FloydWarshall}[]{
	\qinput{}
	\qoutput{}
}
$d$ \qlet ($|V| \times |V|$)-matrix\\
\qfor $k = 1$ \qto $|V|$\\
	\qfor $(i, j) \in |V| \times |V|$\\
		$d_{i, j}$ \qlet $\min (d_{i, j}, d_{i, k} + d_{k, j})$
	\qrof
\qrof
\end{algorithm}

\paragraph{Laufzeit}
Die Laufzeit des Algorithmus liegt in $\mathcal{O}(|V|^3)$.
Anhand der Darstellung als Adjazenzmatrix, die alle möglichen Paare $(i, j)$ in $|V| \times |V|$ enthält,
über die $|V|$ Mal iteriert wird, lässt sich das leicht nachvollziehen.

\paragraph{Beispiel}
TODO: beispielhafte Ausführung wie in der Klausur anzuwenden

\subsection{Minimal-spannende Bäume}
\label{subsec:MinimalSpannendeBaeume}

\subsubsection{Generischer Algorithmus}

Definition: Ein Schnitt von $G = (V, E)$ ist eine Zerlegung von $V$ in zwei nichtleere Teilmengen $(S, V \setminus S)$.

Satz: Sei $T = (V, E')$ MST von $G = (V, E)$.
Angenommen $e \in E$ ist Kante mit minimalem Gewicht aus $E_{s} = \{uv | u \in S, v \in v\setminus S\}$ für Schnitt $(S, V \setminus S)$,
dann gibt es in $E'$ eine Kante $e' \in E_{S}$, so dass $|e'| \leq |e|$.

\subsubsection{Prim-Jarnik-Dijkstra}
\label{subsubsec:PrimJarnikDijkstra}
Der Algorithmus geht von einem Startknoten $v$ aus,
der anfangs alleine eine Zusammenhangskomponente $C$ bildet.
Diese wird nun schrittweise um diejenigen Knoten erweitert,
zu denen diejenige leichteste Kante führt, die aus $C$ heraus führt.

\begin{algorithm}{PrimJarnikDijkstra}[G]{
	\qinput{}
	\qoutput{}
}
\end{algorithm}

\paragraph{Laufzeit}

\paragraph{Beispiel}

\subsubsection{Kruskal}
\label{subsubsec:Kruskal}
Der Algorithmus betrachtet anfangs jeden Knoten als eine eigene Zusammenhangskomponente.
Er fügt dann solange jeweils diejenigen leichtesten Kanten ein,
die zwei unterschiedliche Zusammenhangskomponenten verbinden,
bis es nur noch eine Zusammenhangskomponente gibt.

\begin{algorithm}{Kruskal}[G]{
	\qinput{}
	\qoutput{}
}
\end{algorithm}

\paragraph{Laufzeit}

\paragraph{Beispiel}

\subsubsection{Borouvka}
\label{subsubsec:Borouvka}
Der Algorithmus iteriert über die momentan vorhandenen Zusammenhangskomponenten.
Für jede wählt er diejenige leichteste Kante aus, die aus ihr heraus führt.
Anschließend werden nun verbundene Zusammenhangskomponenten zusammen gefügt und erneut über diese iteriert,
solange, bis nur noch eine Zusammenhangskomponente besteht.

\begin{algorithm}{Borouvka}[G]{
	\qinput{}
	\qoutput{}
}
\end{algorithm}

\paragraph{Laufzeit}

\paragraph{Beispiel}


\section{Anwendungen}
\label{sec:Anwendungen}

\subsection{Softwareentwurf}
\label{subsec:Softwareentwurf}


\subsection{Arbeiten mit Zeichenketten}
\label{subsec:ArbeitenMitZeichenketten}

\subsubsection{Huffman-Kodierung}
\label{subsubsec:HuffmanKodierung}

Allgemeines Bla-Bla.

Ein Huffman-Baum ist ein binärer Baum.
Darin hat jeder Knoten entweder genau zwei Knoten, oder genau ein Zeichen als Kinder.

\paragraph{Aufbau eines Huffman-Baumes}

Um einen Huffman-Baum aufzubauen muss zuerst die Häufigkeit jedes Zeichens ermittelt werden.
Für jedes der Zeichen wird dann ein Knoten mit der Häufigkeit des Zeichens als Wert erzeugt,
der das Zeichen als Kind hat und selbst bereits einen Huffman-Baum darstellt.

All diese Bäume werden nun in eine Prioritätswarteschlange eingefügt,
die nach dem Wert in der Wurzel eines Baumes sortiert.
Aus dieser Warteschlange werden nun solange die beiden kleinsten Bäume entnommen,
bis die Warteschlange nur noch ein Element enthält.
Die beiden entnommenen Bäume werden Kinder eines neuen Baumes,
dessen Wurzel die Summe der Werte in den Wurzeln der beiden Bäume als Wert hat.
Dieser Baum wird wiederum in die Warteschlange eingefügt.

Zuletzt bleibt der gesuchte Huffman-Baum als Einziger in der Warteschlange zurück.

\paragraph{Kodierung}

\paragraph{Dekodierung}

\paragraph{Beispiel}

Die verschiedenen Schritte der Huffman-Kompression wollen wir nun an einem Beispiel nachvollziehen.
Dazu soll das Wort "`RELIEFPFEILER"' (4 $\times$ E; 2 $\times$ R, L, I, F; 1 $\times$ P) kodiert werden.
Aus den Häufigkeiten der Zeichen können wir folgende Huffman-Bäume erzeugen:

\begin{figure}[h]%[htbp]
%\includegraphics{huffmanbaeume.png}
\caption{Huffman-Bäume}
\label{fig:HuffmanBaeume}
\end{figure}

Diese verschmelzen wir nun mithilfe einer Prioritätswarteschlange zu unserem fertigen Huffman-Baum:

\begin{figure}[h]%[htbp]
%\includegraphics{verschmelzenvonhuffmanbaeumen.png}
\caption{Verschmelzen von Huffman-Bäumen}
\label{fig:VerschmelzenVonHuffmanbaeumen}
\end{figure}

Dieser Baum enthält nun für jedes Zeichen die entsprechende Kodierung:

\begin{figure}[h]%[htbp]
\begin{center}
\begin{tabular}{c|l|l}
Zeichen	& Häufigkeit	& Kodierung	\\\hline
R	& 2		& 00		\\
P	& 1		& 010		\\
F	& 2		& 011		\\
E	& 4		& 10		\\
I	& 2		& 110		\\
L	& 2		& 111
\end{tabular}
\end{center}
\caption{Kodierungen}
\label{fig:KodierungenHuffman}
\end{figure}

Es ist zu erkennen, dass keine Kodierung Präfix einer anderen Kodierung ist.
Außerdem ist die Kodierung für das häufigste Zeichen (E) am kürzesten.
Bei der geringen Anzahl an Zeichen ist dieser Unterscheid allerdings nocht nicht besonders groß
und auch andere Zeichen (R) haben eine entsprechend kurze Kodierung.

\subsubsection{Rabin-Karp}
\label{subsubsection:RabinKarp}

% Download von Veranstaltungsseite (stringsearch.txt)
\begin{verbatim}
Gegeben: Zwei Zeichenketten
           s = s[1]s[2]...s[k]
	   t = t[1]t[2]...t[l]
wobei l <= k.

Frage: Ist t in s enthalten? Wenn ja, wo ist das erste Vorkommen?

Naiver Algorithmus:
for i := 1 to k - l + 1
  // Does s contain t at position i?
  j <- 1
  while (j <= k and s[i+j-1] = t[j])
    j++
  // If yes, return postion i
  if j = k+1 then
    return i
// Not found, return -1
return -1

Die Laufzeit ist O(kl), was nur akzeptabel ist,  
wenn l klein ist. Kann man die Laufzeit verbessern?

Die Idee von Rabin-Karp: Der Flaschenhals darin, dass wir
innerhalb der for-Schleife jedesmal den kompletten Substring
s[i...i+l-1] mit t vergleichen. Wir koennten die Schleife
beschleunigen, wenn wir vorher einen schnellen Test haetten,
der uns zeigt, ob es wahrscheinlich ist, dass s[i...i+l-1]
und t gleich sind. Dieser Test laesst sich durch eine 
*Hashfunktion* bewerkstelligen. Eine solche Hashfunktion
weist sowohl s[i...i+l-1] und t eine Zahl zu, und diese
Zahlen lassen sich schnell vergleichen. Ausserdem hat eine
gute Hashfunktion wenig Kollisionen, so dass es selten vorkommt,
dass s[i...i+l-1] und t denselben Hashwert erhalten, obwohl
sie unterschiedlich sind.

Dies fuehrt zu folgendem Ansatz (Rabin-Karp-Algorithmus).
Sei h eine Hashfunktion:
for i := 1 to k - l + 1
  if h(s[i...i+l-1]) = h(t) then
    if s[i...i+l-1] = t then
      return i
return -1

Wenn Kollisionen selten sind, sollte der Aufwand fuer den Vergleich
s[i...i+l-1] ?= t vernachlaessigbar sein.
Aber wir haben ein neues Problem: Wie berechnet man die Hashfunktion schnell?
(Ansonsten waere die ganze Idee natuerlich witzlos.)

Zur Erinnerung: Vor ein paar Monaten hatten wir eine Hashfunktion h' fuer einen 
String a = a[0]a[1]...a[l-1] folgendermassen definiert: Wir hatten die Zeichen 
a[i] als Zahlen zwischen 0 und S-1 interpretiert (S = |Sigma|, die 
Alphabetgroesse), und geschrieben
  
  h'(a) = (sum_{j=0}^{l-1} a[j]*S^{j}) mod p

Fuer Rabin-Karp ist es besser, die Hashfunktion etwas anders zu definieren:

  h(s[i...i+l-1]) = (sum_{j=0}^{l-1} s[i+j]*S^{l-1-j}) mod p

h unterscheidet sich von h' in zwei Punkten: 
  1. wir addieren zum Index im String immer i (weil es sich um einen 
     Teilstring handelt), und 
  2. die Potenzen von S sind jetzt fallend statt steigend (das macht 
    die Formel unten angenehmer).

Der springende Punkt ist nun die folgende Beziehung:

  h(s[i+1...i+l]) = (S*h(s[i...i+l-1])-s[i]*S^l+s[i+l]) mod p

Das heisst: 
  *wenn wir h(s[i...i+l-1]) kennen, koennen wir
  h(s[i+1...i+l]) mit O(1) Operationen berechnen.*
(beachte: S^l muessen wir nur einmal ausrechenen und speichern).

Daraus folgt: wir koennen h(s[1...l]), h(s[2...l+1]), ..., h(s[k-l, l)] sowie
h(t) in Gesamtzeit O(k+l) berechnen, und heuristisch gesehen sollte der
Algorithmus von Rabin-Karp nun O(k+l) Zeit benoetigen, da wir hoffen, 
dass Kollisionen selten sind. (Genauer besteht die Idee fuer die 
Laufzeitanalyze darin, p = Theta(l) zu waehlen. Dann sollte die
Wahrscheinlichkeit einer Kollision etwa 1/l betragen, so dass man 
nur in jedem l. Schleifendurchlauf den Stringvergleich durchfuehren
muss, was zu konstanter amortisierter Zeit pro Schleifendurchlauf
fuehrt.)

Bemerkungen:
  1. Wenn man den Algorithmus implementiert, sollte man bei der Berechnung
     von h nach jeder Multiplikation das Ergebnis (mod p) nehmen und sicher
     stellen, dass p^2 nicht zu gross ist, um Ueberlaeufe zu vermeiden.
  2. Wenn man p geeignet zufaellig waehlt, kann man erreichen, dass der
     Algorithmus von Rabin-Karp O(k+l) Zeit im Erwartungswert benoetigt.
  3. Es gibt andere Algorithmen zur Suche in Zeichenketten, die O(k+l)
     worst-case-Zeit erreichen (Knuth-Morris-Pratt, Boyer-Moore, Suffix-Baeume).
     Diese sind aber zT bedeutend komplizierter.
  4. Die Idee, Strings durch Hashfunktionen darzustellen/zu approximieren ist
     sehr fruchtbar und hat viele Anwendungen (sichere Speicherung von Passwoertern,
     digitale Unterschriften, Verifikation von Downloads, ...)
\end{verbatim}

\subsection{Spieltheorie}


\chapter{Nichtsequentielle Programmierung}
\chapter{Netzprogrammierung}


%\part{Mathematik für Informatiker}
%\part{Mathematik für Informatiker}
\chapter{Logik und diskrete Mathematik}
%
%
%
\section{Boolesche Aussagenlogik}

\subsection{Grundbegriffe; Vom Booleschen Term zur Booleschen Funktion}
\subsection{Von der Booleschen Funktion zum Booleschen Term}
\subsection{Der Gebrauch von Quantoren}
%
%
%
\section{Einführung Mengenlehre}
%
%
%
\section{Relationen und Funktionen}

\subsection{Grundbegriffe}
\subsection{Äquivalenzrelationen}
\subsection{Halbordnungsrelationen und totale Ordnungen}
\subsection{Funktionen}
\subsection{Abzählbarkeit}
%
%
%
\section{Mathematische Beweise; Vollständige Induktion}

\subsection{Das Schubfachprinzip von Dirichlet}
\subsection{Prinzipielles zu mathematischen Beweisen}
\subsection{Natürliche Zahlen und das Prinzip der vollständigen Induktion}
%
%
%
\section{Kombinatorik}

\subsection{Abzählen I}
\subsubsection{Binomialkoeffizienten}
\subsubsection{Binomialkoeffizienten und monotone Gitterwege}
\subsubsection{Mengenpartitionen}
\subsubsection{Zahlpartitionen}
\subsubsection{Doppeltes Abzählen}
\subsection{Die 12 Arten des Abzählens und ein Kartentrick}
\subsection{Diskrete Wahrscheinlichkeitsrechnung; Grundlagen}
\subsection{Erwartungswert, Spezielle Verteilungen}
\subsection{Das Coupon-Collector-Problem}
\subsection{Ein Random Walk und eine randomisierte Strategie}
\subsection{Abzählen III: Lineare Rekursionsgleichungen}
%
%
%
\section{Graphentheorie}

\subsection{Einführung und Grundlagen}
\subsubsection{Beispiele für algorithmische Aufgabenstellungen}
\subsubsection{Grundlegende Begriffe}
\subsection{Zusammenhang und Abstand in ungerichteten Graphen}
\subsection{Charakterisierung bipartiter Graphen}
\subsection{Bäume und ihre Charakterisierung}
\subsection{Grundlegende graphentheoretische Algorithmen}
\subsubsection{Graphdurchmustern: Breitensuche und Tiefensuche}
\subsubsection{Gerichtete azyklische Graphen}
\subsubsection{Einfache Anwendungen von Breiten- und Tiefsuche}
\subsection{Das Minimum-Spanning-Tree Problem: MST}
\subsubsection{Der MST-Algorithmus von Prim}
\subsubsection{Der MST-Algorithmus von Kruskal}
\subsection{Die Euler-Formel für planare Graphen; Maximales Matching}
\subsubsection{Die Euler-Formel}
\subsubsection{Reguläre Polyeder}
\subsubsection{Der Heiratssatz: Maximales Matching in bipartiten Graphen}
%
%
%
\section{Logik II}
\subsection{Der Resolutionskalkül}
\subsection{Hornformel und Einheitsresolventen}
\subsection{Algebraische Strukturen und Prädikatenlogik}

\chapter{Analysis}
\section{Aufbau des Zahlensystems I}
\subsection{Natürliche Zahlen, Peanosches Axiomensystem}
\subsection{Ganze Zahlen}
\subsection{Halbgruppe, Gruppe, Ring, Elementare Eigenschaften}
\subsection{Rationale Zahlen als Aquivalenzklassen, Körper}
\subsection{Reelle Zahl als unendliche Dezimalbrüche und als Aquivalenzklasse von Cauchy-Folgen}
\subsection{algebraische und transzendente reelle Zahlen}
\subsection{Ordnungseigenschaften reeller Zahlen, Begriffe: lineare Ordnung, Schnitt, Schranke, Min/Max, Inf/Sup}
\subsection{Betragsfunktion, Rechnen mit Ungleichungen, Bernoulli-Ungleichung, Ungleichung von Cauchy-Schwarz}

\section{Der Polynomring R[x]}
\subsection{Rechenregeln, Ringstruktur}
\subsection{Effektives Polynomauswerten mittels Horner-Schema}
\subsection{Nullstellen, Linearfaktoren, Faktorisierung reeller Polynome}
\subsection{Bestimmung rationaler Nullstellen bei rationalen Polynomen}
\subsection{Ganz rationale, echt gebrochen rationale Funktionen}
\subsection{Polynomdivision, Euklidscher Algorithmus}
\subsection{Interpolation mit Polynomen, Lagrange-Interpolation, Newton-Interpolation}
\subsection{Eine informatische Anwendungen: Karp-Rabin-Fingerprint zum String Matching}

\section{Aufbau des Zahlensystems II: Komplexe Zahlen}
\subsection{komplexe Zahlenebene}
\subsection{Rechnen mit komplexen Zahlen, Körperstruktur}
\subsection{Betragsfunktion, konjugiert komplexe Zahl}
\subsection{komplexe Polynome, Fundamentalsatz der Algebra}
\subsection{Faktorisierung reeller Polynome in lineare und quadratische Terme}
\subsection{Polarkoordinaten, komplexe Exponentialfunktion}
\subsection{Formeln von De Moivre, Rechnen in Polardarstellung, Potenzieren, Radizieren}
\subsection{Einheitswurzeln}
\subsection{Historisch: Lösung kubischer Gleichungen}
\subsection{Überlagerung von Schwingungen}

\section{Folgen und Grenzwerte}
\subsection{Begriff der Folge, Beispiele, Konvergenz/Divergenz, Eindeutigkeit des Grenzwertes}
\subsection{Konvergenzkriterien: Vergleichskriterium, Cauchy, beschränkte monotone Folgen}
\subsection{Rechenregeln für Grenzwerte}
\subsection{Folge der Partialsummen (Reihe)}
\subsection{Geometrische Folgen und Reihen, Anwendung: Koch’sche Eisblume}
\subsection{harmonische Reihe und alternierende harmonische Reihe}
\subsection{Eulersche Zahl als Grenzwert}
\subsection{O-Notation: Asymptotisches Wachstum, $\mathcal{O}(f(n))$, $\Omega(f (n))$, $\Theta(f (n))$, $o(f(n))$, $\omega(f (n))$, Rechenregeln, Wachstum von Standardfunktionen, Stirling-Formel}

\section{Stetigkeit von Funktionen, Differentiation}
\subsection{Grenzwert einer Funktion in einem Punkt, einseitige Grenzwerte, Beispiele}
\subsection{Grenzwertarithmetik für Funktionen, das Einschnüren von Funktionstermen (Bsp. $\lim \sin x/x)$}
\subsection{Asymptoten des Funktionsgraphen, Polstellen gebrochen rationaler Funktionen}
\subsection{Stetigkeit über einem Intervall, $\delta$-Kriterium}
\subsection{Stetige Fortsetzung, gleichmäßige Stetigkeit}
\subsection{Eigenschaften stetiger Funktionen: Komposition stetiger Funktionen ist stetig, Nullstellensatz, Zwischenwertsatz, Min-Max-Eigenschaft}
\subsection{Ableitung in einem Punkt, Differenzierbare Funktionen sind stetig}
\subsection{Geometrische Interpretation: Anstieg der Tangente, Analytische Interpretation:
Lineare Approximation}
\subsection{Rechenregeln zum Differenzieren, höhere Ableitungen}
\subsection{Ableitung trigonometrischer Funktionen}
\subsection{Extremalstellen einer Kurve, Mittelwertsatz und seine Verallgemeinerung}
\subsection{Bestimmung Extremalstellen und Wendepunkte, Kurvendiskussion}
\subsection{Regel von L’Hospital}
\subsection{Bestimmung von Nullstellen/Fixpunkten: direkte Fixpunktiteration, Newton-Verfahren}
\subsection{Taylorapproximation I: Taylor-Polynom n-ten Grades, Lagranges Restglied, Beispiele}
\subsection{Umkehrfunktionen und ihre Ableitung, Spezielle Funktionen: Wurzelfunktion, Umkehrfunktion der trigonometrischen Funktion, Exponential- und Logarithmusfunktion}
\subsection{Hyperbelfunktionen}

\section{Integration}
\subsection{Bestimmtes Integral: Unter- und Obersummen, Eigenschaften, Riemannsches Integral, Beispiele}
\subsection{Eigenschaften des Riemann-Integrals: Monotonie, Linearität}
\subsection{Riemannsches Kriterium zur Integrierbarkeit, beschränkte monotone/stetige Funktionen sind integrierbar}
\subsection{Mittelwertsatz der Integralrechnung}
\subsection{Anwendung: Volumenberechnung bei Rotationskörpern}
\subsection{Begriff der Stammfunktion, Hauptsatz der Differential- und Integralrechnung}
\subsection{unbestimmtes Integral, Beispiele}
\subsection{Partielle Integration, Substitutionsregel, Partialbruchzerlegung, Beispiele}
\subsection{Uneigentliche Integrale}

\section{Potenzreihen}
\subsection{Reihen, Beispiele, Rechenregeln, Konvergenz, Majorantenkriterium}
\subsection{Reihen von Funktionen:
Grenzfunktion
Konvergenzkriterien:
Stetigkeit
und
Cauchy,
Differentiation/Integration}
\subsection{Potenzreihen, Konvergenzradius und dessen Berechnung}
\subsection{Potenzreihendarstellung einiger Standardfunktionen, Binomialreihe}
\subsection{Taylorreihe einer Funktion, Anwendungen}

\chapter{Lineare Algebra}
%
%
%
\section{Lineare Algebra}
\subsection{Anschauliche Vektorrechnung}
\subsection{Gruppen und Körper}
\subsection{Vektorräume}
\subsection{Lineare Unabhängigkeit, Basis und Dimension}
\subsection{Lineare Abbildungen}
\subsection{Matrizen}
\subsection{Der Rang einer Matrix}
\subsection{Lineare Gleichungssysteme}
\subsection{Inverse Matrizen}
\subsection{Determinanten}
\subsection{Euklidische Vektorräume}
\subsection{Eigenwerte und Eigenvektoren}
%
%
%
\section{Endliche Körper und Lineare Codes}
\subsection{Endliche Körper}

Ein Endlicher Körper $\mathbb{F}_p$ oder $GF(p)$ ist ein Körper der Restklassen ganzer Zahlen bezüglich einer Primzahl $p$.

\subsection{Fehlererkennung und Fehlerkorrektur in Codes}
\subsection{Allgemeine Schranken für die Informationsrate}
\subsection{Lineare Codes}
%
%
%
\section{Stochastik}
\subsection{Wahrscheinlichkeitsräume}

Ein Wahrscheinlichkeitsraum besteht aus: ?

\subsection{Bedingte Wahrscheinlichkeit und Unabhängigkeit}
\subsection{Zufallsvariable}
\subsection{Erwartungswerte}
\subsection{Abweichungen vom Erwartungswert}



\part{Technische Informatik}
\part{Technische Informatik}
\chapter{Grundlagen der Technischen Informatik}
\chapter{Rechnerarchitektur}
\chapter{Betriebs- und Kommunikationssysteme}
\section{Notizen aus dem letzten Tutorium}
\begin{verbatim}
Kernel, monolithisch vs. micro, Linux ansehen
Interrupts, Systemaufrufe, Kontextwechsel
Prozess vs. Thread
PCB, Organisation von Prozessen durch das OS, Verwaltung durch Queues
Memory Management, Freispeicherverwaltung
drei verschiedene Adressierungen: absolute, relative, virtuelle Adressen
Fragmentierung
Caching
Scheduling (preemptive, non-preemptive)
Netzwerke Layer 1 3 4 7
\end{verbatim}
%
%
%
\section{Betriebssysteme}

\subsection{Schichtenmodell}

In welche sieben Ebenen kann man ein Rechnersystem einteilen?
\cite{kreissl}
\begin{enumerate}
\item Anwendungsebene (Anwendersoftware)
\item Assemblerebene (Beschreibung von Algorithmen, Link \& Bind)
\item Betriebssystem (Speichermanagment, Prozesskommunikation)
\item Instruction Set Architecture (ISA, Adressierungsarten)
\item Microarchitektur (Risc, Cisc, Branch Prediction..)
\item Logische Ebene (Register, Schieber, Latches..)
\item Transistorebene (Transistoren, MOS)
\end{enumerate}

Aufgabe eines Betriebssystems ist die Verwaltung von
Speicher,
Prozessen (Scheduling, Sicherheit) und
Hardwareressourcen (I/O).

\subsection{Speicherverwaltung}

\subsubsection{Aufteilung (Paging)}
Verschiedene Verfahren, den verfügbaren Speicher aufzuteilen:
Feste Speicherblöcke (Fixed Partitions), Problem: Interne Fragmentierung
Speicherblöcke variabler Größe (Dynamic Partitions), Problem: Externe Fragmentierung
Dynamische Algorithmen (First-Fit, Next-Fit)
Buddy System (?)
Paging

\subsubsection{Verwaltung (Swapping, Replacement-Policies)}
LRU, FIFO, Clock, (Optimal)

\subsection{Dateisystem}

\subsection{Prozessverwaltung}

\subsubsection{Scheduling}

\subsection{I/O-Verwaltung}

Programmed I/O, busy waiting\\
Interrupt-driven I/O\\
Direct Memory Access (DMA)

%
%
%
\section{Kommunikationssysteme}

\subsection{Schichtenmodelle (OSI und TP/IP)}

Der Aufbau von Computernetzwerken wird durch verschiedene Modelle beschrieben.
Eines davon ist das OSI-Modell (Open Systems Interconnection Reference Model),
welches eine sehr feine Unterschiedung verschiedener Aufgaben bei der Netzwerkkommunikation macht.
Ein anderes ist das TCP/IP-Referenzmodell,
welches im heutigen Aufbau des Internets umgesetzt ist und einige Schichten des OSI-Modells zusammenfasst:

\paragraph{Anwendungsschicht}
OSI: Anwendung (Application), Darstellung (Presentation), Sitzung (Session)
\paragraph{Transportschicht}
OSI: Transport
\paragraph{Internetschicht}
OSI: Vermittlung (Network)
\paragraph{Netzzugangsschicht}
OSI: Bitübertragung (Physical), Sicherungsschicht (Data Link)
Mit dieser Schicht beschäftigt sich das TCP/IP-Modell nicht näher.
Behandelt werden sollen Ethernet, WLAN und Bluetooth.

\subsection{Netzzugangsschicht}
Aufgabe dieser Schicht ist es, die Übertragung von Signalen zu ermöglichen.
Signale sind die Veränderung einer physikalischen Größe über einer bestimmten Zeit.
Sie stellen die Abstraktion von Daten dar.
Zu dieser Übertragung wird ein Medium benötigt, über das die Signale übertragen werden.
Dieses können zum Beispiel verschiedene Arten von Kabeln sein,
oder auch weniger anschauliche Medien wie der leere Raum bei drahtloser Übertragung.


\subsection{Internetschicht}

\subsubsection{IP-Header}


\subsection{Transportschicht}

\subsubsection{TCP- und UDP-Header}


\subsection{Anwendungsschicht}

\subsection{Eine Reise durch das Internet}

Was passiert, wenn ich eine Seite im Browser öffne, bzw. eine E-Mail versende?



\part{Theoretische und Praktische Informatik}
\part{Theoretische und Praktische Informatik}
\chapter{Grundlagen der Theoretischen Informatik}
%
%
%
\section{Einführung}

\chapter{Proseminar}
\chapter{Datenbanksysteme}
%1. Datenmodellierung
%systematischer Entwurf von DB
%Schwerpunkt: Relationale DB
%2. Datenbanknutzung
%Zugriff auf die Daten mit SQL (Structured Query Language), interaktiv oder mit Anwendungsprogrammen.
%
%3. Implementierungsaspekte von DBS
%Transaktionen, Synchronisation, (Indexierung)
%
%4. Einführung in neuere Techniken der Datenverwaltung
%Data Warehouse (OLAP), Information Retrieval, Data Mining

%
%
%
\section{Einführung}

Eine \textbf{Datenbank} ist eine Menge von Datenobjekten,
die einem bestimmten \textit{Datenschema} genügen.

Ein \textbf{Datenbankschema} ist eine formale Beschreibung eines Teils der Wirklichkeit
mit den Begrifen eines bestimmten \textit{Datenmodells} (z.B. Tabellen und Klassen).

Ein \textbf{Datenmodell} ist eine Sprache um ein
\textit{Datenbankschema} (mittels der Data Definition Language) zu beschreiben,
sowie auf eine \textit{Datenbank} (mittels der Data Manipulation Language) zuzugreifen und sie zu bearbeiten.

%
%
%
\section{Thoretische Grundlagen}

\subsection{Schlüssel}

Ein \textbf{Superschlüssel} ist eine Menge von Attributen einer Relation,
für die keine zwei Tupel die gleichen Werte haben.
Jede Relation hat mindestens einen Superschlüssel.

% Was ist das?s
$t_1$, $t_2$ in the relation instance $r(R)$
$t_1$ [SK] $\neq$ $t_2$ [SK]

Ein \textbf{Schlüssel} $K$ einer Relation $R$ ist ein Superschlüssel von $R$,
von dem kein Attribut $\alpha$ entfernt werden kann,
ohne dass $K \setminus \alpha$ kein Superschlüssel von $R$ mehr ist.

Schlüssel sollen aus der Bedeutung der Attribute hergeleitet und nicht
aus den vorhandenen Werten der einzelnen Instanzen abgeleitet werden.
Eine Relation kann auch mehrere Schlüssel haben, die dann \textbf{Schlüsselkandidaten} heißen.
Von ihnen wird einer zum \textbf{Primärschlüssel} bestimmt, der dann verwendet wird,
um die Tupel der Relation eindeutig zu identifizieren.

Um in einer Relation Bezug auf einen Datensatz einer anderen Relation zu nehmen,
wird dessen Primärschlüssel $PK$ als sogenannter \textbf{Fremdschlüssel} $FK$ gespeichert.

Für einen Fremdschlüssel $FK$ in einer Relation $R_1$ muss gelten,
dass die Attribute in $FK$ die gleiche Bedeutung haben, wie in $PK$.

% Wie übersetzt man das sinnerhellend?
A value of FK in t1 in R1 :
either occurs as a value of PK for t2 in R2
i.e., t1 [FK] = t2 [PK]
or is null.


%
%
\subsection{Modellierung}

\subsubsection{Chen-Notation}

\paragraph{Entitäten}

(Rechteck),
schwache Entität (doppeltes Rechteck)

\paragraph{Relationen}

\paragraph{Attribute}

Attribute beschreiben Eigenschaften von Entitäten und Relationen.
Mn unterscheidet \textbf{einfache Attribute} (Oval, verbunden mit Entität),
\textbf{mehrwertige Attribute} (doppeltes Oval, verbunden mit Entität) und
\textbf{zusammengesetzte Attribute} (Oval, verbunden mit Attribut).
\textbf{Abgeleitete Attribute} (gestricheltes Oval).

\paragraph{}

\subsubsection{(min, max)-Notation}

%
%
\subsection{Funktionale Abhängigkeiten}

Funktionale Abhängigkeiten drücken Beziehungen zwischen Attributen einer Relation aus.
Eine Funktionale Abhängigkeit $F$ der Form $A \rightarrow B$ drückt dabei aus, dass das Attribut $A$ als Determinante von $F$ das Attribut $B$ eindeutig bestimmt.

\subsubsection{Armstrong-Axiome}

\begin{enumerate}
\item Reflexivität:\\ $X \subseteq Y \Rightarrow Y \rightarrow X$
\item Erweiterung / Verstärkung:\\ $X \rightarrow Y \Rightarrow XZ \rightarrow YZ$
\item Transitivität:\\ $X \rightarrow Y \land Y \rightarrow Z \Rightarrow X \rightarrow Z$
\item Vereinigung:\\ $X \rightarrow Y \land X \rightarrow Z \Rightarrow X \rightarrow YZ$
\item Dekomposition:\\ $X \rightarrow YZ \Rightarrow X \rightarrow Y \land X \rightarrow Z$
\item Pseudo-Transitivität:\\ $X \rightarrow Y \land YZ \rightarrow W \Rightarrow XZ \rightarrow W$
\end{enumerate}

\subsubsection{?}

Minimal cover of a set of FDs, Closure of attribute set

\subsubsection{kanonische Überdeckung}

Aus Wikipedia\footnote{\url{http://de.wikipedia.org/wiki/Kanonische_\%C3\%9Cberdeckung}}

Zu einer gegebenen Menge $F$ von funktionalen Abhängigkeiten nennt man $F_*$ eine kanonische Überdeckung, wenn folgende drei Eigenschaften erfüllt sind:
\begin{enumerate}
\item $F_* \equiv F$, das heißt $F_*^+ = F^+$
\item In $F_*$ existieren keine funktionalen Abhängigkeiten $\alpha \rightarrow \beta$, wobei $\alpha$ und $\beta$ Mengen überflüssiger Attribute enthalten; das heißt, es muss gelten:
\begin{enumerate}
\item $\forall A \in \mathbf \alpha: (F_* - (\alpha \rightarrow \beta) \cup ((\alpha - A) \rightarrow \beta)) \not\equiv F_*$
\item $\forall B \in \mathbf \beta: (F_*- (\alpha \rightarrow \beta) \cup (\alpha \rightarrow (\beta-B))) \not\equiv F_*$
\end{enumerate}
\item Jede linke Seite einer funktionalen Abhängigkeit in $F_*$ ist einzigartig. Dies kann durch fortgesetzte Anwendung der Vereinigungsregel auf funktionale Abhängigkeiten der Art $\alpha \rightarrow \beta$ und $\alpha \rightarrow \gamma$ erreicht werden, so dass die beiden funktionalen Abhängigkeiten durch $\alpha \rightarrow \beta\gamma$ ersetzt werden.
\end{enumerate}

\begin{enumerate}
\item Linksreduktion
\item Rechtsreduktion
\item Alle funktionalen Abhängigkeiten der Form $\alpha \rightarrow \{\}$ entfernen
\item Alle funktionalen Abhängigkeiten $\alpha \rightarrow \beta$ aus $F$ mit gleichem $\alpha$ zusammenfassen: Wenn $\alpha \rightarrow \beta \in F$, $\alpha \rightarrow \gamma \in F$, dann entferne diese beiden funktionalen Abhängigkeiten aus $F$ und füge $\alpha \rightarrow \beta\gamma$ zu $F$ hinzu.
\end{enumerate}

\subsubsection{Attributhülle}
Die Attributhülle $A^+$ eines Attributes $A$ bezeichnet die Menge all derjenigen Attribute, die von $A$ funktional abhängig sind.

\subsubsection{abgeschlossene Hülle}
Die abgeschlossene Hülle $F^+$ einer Menge von funktionalen Abhängigkeiten $F$ bezeichnet die Menge all derjenigen Attribute, die sich von den funktionalen Abhängigkeiten aus $F$ und deren Determinanten bestimmen lassen.

%
%
\subsection{Normalisierung}

\subsubsection{Erste Normalform (1. NF)}

Voraussetzung für die erste Normalform ist, dass alle Attribute atomar sind.
What we called relation so far.

Nicht in erster Normalform (N1NF) sind geschachtelte Relationen, in denen Attribute wiederum Relationen sind.

\subsubsection{Zweite Normalform (2. NF)}

Historical.

\subsubsection{Dritte Normalform (3. NF)}

Ein relationales Schema $R$ ist in 3. NF wenn, immer wenn eine funktionale Abhängigkeit X $\rightarrow$ An holds in R, entweder X ein Superschlüssel von R ist, oder A ein Primattribut von R ist.

Theorem:
Jede Relation in 1. NF besitzt eine ”well-behaved“ Dekomposition, i.e., in 3. NF, with lossless join that preserves dependencies.

Assume F is a minimal cover.
Algorithm to achieve a well-behaved 3NF decomposition.

1. For each FD (X $\rightarrow$ A) in F create a relation with schema (XA).

2. If none of the keys appears in one of the schemas of 1 then add a relation with schema Y, with Y a key.

3. If for relations created in 1. there exists a relation R1 whose schema is included in the schema of another relation, then remove R1.

4. Replace relations (X A1), ..., (X Ak) with a single relation (X A1 ... Ak).

\subsubsection{Boyce-Codd-Normalform (BCNF)}

A relation is in BCNF if for each $X \rightarrow A$ in F+, X is a superkey.

Theorem:
BCNF $\Rightarrow$ 3NF $\Rightarrow$ 2NF.

\subsubsection{Vierte Normalform (4. NF)}

Später, nicht Klausur-relevant.

%
%
%
\section{Sprachen zur Abfrage und Manipulation von Datenbanken}

%
%
\subsection{Relationale Algebra}

\subsubsection{Vereinigung $\cup$}

\subsubsection{Schnittmenge $\cap$}

\subsubsection{Differenz $\setminus$}

\subsubsection{Kartesisches Produkt $\times$}

\subsubsection{Projektion $\pi$}

\subsubsection{Selektion $\sigma$}

\subsubsection{Join $\bowtie$}

%
%
\subsection{Structured Query Language (SQL)}

Structured Query Language (SQL) ist eine formale Sprache, die Möglichkeiten zur Anfrage und Manipulation von Datenbanken auf verschiedenen Ebenen bietet.

Auf der strukturellen Ebene können innerhalb eines DBMS ganze Datenbanken und innerhalb einer Datenbank Tabellen erstellt, verändert oder gelöscht werden.

Auf der inhaltlichen Ebene ermöglicht SQL Datensätze zu lesen, bearbeiten und zu löschen.

\subsubsection{Strukturelle Operationen}

CREATE / CHANGE / DROP DATABASE
CREATE / CHANGE / DROP TABLE

\subsubsection{Inhaltliche Operationen}

SELECT
UPDATE
DELETE

Aggregatfunktionen
SUM, AVG, MAX, MIN, COUNT
DISTINCT
GROUP BY
UNION

GROUPING SETS
CUBE
ROLLUP

\subsubsection{Datentypen}

DATE, INT, VARCHAR, TEXT

%
%
%
\section{Technische Umsetzung}

%
%
\subsection{Verarbeitung und Optimierung von Anfragen}

Schritte der Verarbeitung

\begin{enumerate}
\item Query in a high-level query language\\
SCANNING, PARSING, VALIDATING
\item Intermediate form of a query\\
QUERY OPTIMIZER
\item Execution plan\\
QUERY CODE GENERATOR
\item Code to execute the query (interpreted or compiled)\\
RUNTIME DB PROCESSOR
\item Result of query

\end{enumerate}

\subsubsection{SCANNING, PARSING, VALIDATING}

\subsubsection{QUERY OPTIMIZER}

Query-Trees
Perform selection operation as early as possible.
Replace expressions of the form
$\sigma P1 \land P2 (e)$
with
$\sigma P1 (\sigma P2 (e))$
Perform projections early

\subsubsection{QUERY CODE GENERATOR}

\subsubsection{RUNTIME DB PROCESSOR}

%
%
\subsection{Transaktionen}

%
%
\subsection{Nebenläufigkeit}

\chapter{XML-Technologien}
%
%
%
\section{Modellierung und Validierung}

\subsection{Documenttypdefinition}

Definition von XML-Dokumenten über Elemente, Attribute und Entitäten.
http://w3schools.com/dtd/

\subsection{XML Schema Definitionen}


\section{Abfrage von XML-Daten}

\subsection{XPath}
\subsection{XQuery}


\section{Formatierung und Transformation}

\subsection{Extensible Stylesheet Language (XSL)}

XSLT, XPath, XSL-FO

\chapter{Höhere Algorithmik}
%
%
%
\section{Einführung}

Dieser Kurs behandelt den Entwurf, die Analyse und Bewertung von Algorithmen.

Ein \textbf{Algorithmus} ist ein endlich beschriebenes effektives Verfahren,
das eine Eingabe in eine Ausgabe überführt.

Beispiele für Algorithmen sind etwa Sortieralgorithmen
(Quicksort\index{Algorithmen!Sortieren!Quicksort},
Mergesort\index{Algorithmen!Sortieren!Mergesort},
Heapsort\index{Algorithmen!Sortieren!Heapsort},
...),
Suchalgorithmen (binäre Suche\index{Algorithmen!Suchen!binäre Suche},
Suche in speziellen Datenstrukturen)
oder Graphenalgorithmen
(Kruskal\index{Algorithmen!Graphen!Kruskal},
Floyd-Warshall\index{Algorithmen!Graphen!Floyd-Warshall},
Bellman-Ford\index{Algorithmen!Graphen!Bellman-Ford},
...).
Die meisten dieser genannten Algorithmen würde man jedoch nicht als höhere
Algorithmen bezeichnen. Ein Beispiel soll verdeutlichen, was einen sogenannten
höheren Algorithmus ausmacht.

\subsection{Ein höherer Algorithmus}

Dazu betrachten wir ein Problem aus der Statistik.
Häufig kommt es vor, dass für eine große Datenmenge ein einzelner repräsentativer Wert gefunden werden soll.
Eine Möglichkeit dafür ist der Durchschnitt, der jedoch anfällig für Ausreißer ist.
Eine oftmals bessere Möglichkeit bietet der Median.

Der \textbf{Rang} $rg(s)$ eines Elementes $s \in S$
in einer total geordneten Menge $S$ ist die Anzahl von Elementen in $S$,
die kleiner sind, als $s$.

Der \textbf{Median} einer total geordneten Menge $S$ ist dasjenige Element $s \in S$,
dessen Rang $rg(s) = \ceil{\frac{\abs{S}+1}{2}}$ ist.
Das heißt, es gibt in $S$ genauso viele Elemente, die kleiner sind als $s$,
wie Elemente, die größer sind als $s$.

Eine einfache Möglichkeit den Median zu bestimmen ist,
die Menge zu sortieren und das Element an der Stelle $\ceil{\frac{\abs{S}+1}{2}}$ auszuwählen.
Dieser Ansatz hat eine Laufzeit von $\Theta(n \log n)$.

Ein weiterer Ansatz geht davon aus, dass es eine weitere Funktion $SPLITTER(S)$ gibt,
die zwar nicht genau den Median findet, jedoch eine gute Näherung in Form eines Elementes $q$,
für das gilt $\ceil{\frac{1}{4} n} \leq rg(q) \leq \floor{\frac{3}{4} n}$.

Wenn wir diese Funktion ohne weiteren Aufwand verwenden können, so lässt sich zeigen,
dass wir den Median in linearer Laufzeit bestimmen können.
Dazu lösen wir erst das allgemeinere Auswahlproblem.

\subsection{Das Auswahlproblem}

Finde ein Element $s \in S$ mit $rg(s) = k$.
Idee: Splitter zum Teilen verwenden, rekursiv vorgehen (ähnl. Quicksort)

\begin{algorithm}{SELECT}[S, k]{
}
\qif $\abs{S} < c$\\
	\qthen use brute force\\
	\qelse
		$q$ \qlet SPLITTER($S$)\\
		$S_<$ \qlet $\{s \in S | s < q\}$\\
		$S_>$ \qlet $\{s \in S | s > q\}$\\
		\qif $\abs{S_<} \geq k$\\
			\qthen \qreturn SELECT($S_<$, $k$)\\
			\qelse
				\qif $\abs{S_<} = k-1$\\
					\qthen \qreturn $q$\\
					\qelse \qreturn SELECT($S_>$, $k-\abs{S_<}-1$)
				\qfi
		\qfi
\qfi
\end{algorithm}

Laufzeitanalyse





\begin{algorithm}{SELECT}[S, k]{
}
\qif $\abs{S} < c$\\
	\qthen use brute force\\
	\qelse
		unterteile $S$ in 5er-Gruppen\\
		$S'$ \qlet Mediane aller 5er-Gruppen\\
		$q$ \qlet SELECT($S'$, $\ceil{\frac{\abs{S'}+1}{2}}$)\\
		$S_<$ \qlet $\{s \in S | s < q\}$\\
		$S_>$ \qlet $\{s \in S | s > q\}$\\
		\qif $\abs{S_<} > k$\\
			\qthen \qreturn SELECT($S_<$, $k$)\\
			\qelse
				\qif $\abs{S_<} = k$\\
					\qthen \qreturn $q$\\
					\qelse \qreturn SELECT($S_>$, $k-\abs{S_<}-1$)
				\qfi
		\qfi
\qfi
\end{algorithm}

\subsection{Church-Turing-These}

intuitiv berechenbar = RAM-berechenbar

\subsection{Laufzeit und Speicherplatz}

Gegeben ein RAM-Programm, das eine Funktion $f$ berechnet.
Sei $x \in \mathbb{Z}*$ eine Eingabe. Dann ist
die Laufzeit T(x) die Gesamtkosten der Arbeitsschritte bis das Programm bei Eingabe $x$ hält.
der Speicherplatz S(x) der gesamte Speicherbedarf

\subsubsection{Einheitskostenmaß}
\subsubsection{Logarithmisches Kostenmaß}

Pragmatische Entscheidung: EKM findet Anwendung bei kombinatorischen Algorithmen wie z.B. Sortieren, Suchen, Zeichenketten und Graphen.
LKM bei zahlentheoretischen Algorithmen wie z.B. Primzahltest.

Um allgemeine Aussagen über einen Algorithmus zu erhalten, fassen wir Eingaben $x$ nach ihrer Größe $n = \abs{x}$ zusammen.
Was genau dabei die Größe einer Eingabe darstellt, ist problemspezifisch festzulegen.
Für Zeichenketten bietet sich oft ihre Länge an, für Graphen etwa die Anzahl Knoten oder Kanten.
Um nicht alle verschiedenen Eingaben einer bestimmten Länge testen zu müssen, wird jeweils der worst-case angenommen,
also Laufzeit und Speicherbedarf im jeweils komplexesten Fall berechnet.
Dabei gilt für alle $x$ mit $\abs{x} = n$:
$$T_{worst-case}(n) = \max T(x) \qquad S_{worst-case}(n) = \max S(x)$$

\section{Grundlegende Techniken}

\subsection{Divide et impera}

\subsubsection{Problem des engsten Punktpaares}

Gegeben sind $n$ Punkte in der Ebene $P \subseteq \mathbb{R}^2$, $\abs{P} = n$.
Gesucht sind diejenigen Punkte $p, q \in P, p \neq q$ für die $d(p, q)$ minimal ist.

\subsection{Lösen von Rekursionsgleichungen}

Es gibt verschiedene Ansätze um Rekursionsgleichungen zu lösen und in eine geschlossene Form zu bringen.

\subsubsection{Intuition}
Mit viel Erfahrung oder Glück kann man das Ergebnis erraten und dessen
Richtigkeit anschließend z.B. mittels vollständiger Induktion beweisen.

\subsubsection{Einsetzen} Wiederholt einsetzen und Muster erkennen

\subsubsection{Rekursionsbaummethode} Ein Bild malen und Muster erkennen

\subsubsection{Master-Theorem}
Dies ist ein allgemeines Rezept, das eine große Anzahl von Rekursionsgleichungen
abdeckt, jedoch nicht alle.

Sei $a \geq 1$, $b > 1$ und $f: \mathbb{N} \rightarrow \mathbb{N}$. Sei
$$T(n) = a \cdot T(\frac{n}{b}) + f(n)$$.


\subsection{Dynamisches Programmieren}

\subsubsection{Einkaufsproblem}

\subsubsection{Traveling-Salesman-Problem}


\subsection{Greedy-Algorithmen}

\subsubsection{Münzwechsel}
\subsubsection{Tankstellen-Problem}
\subsubsection{Vorlesungsplanung}


\section{Datenstrukuren}

\subsection{Union-Find}

das Problem der Vereinigung disjunkter Mengen (DSU)

Gegeben ist eine endliche Menge $S = \{1, 2, \ldots, n\}$, sowie eine Partition
von $S$ in nichtleere disjunkte Teilmengen
$S = S_1 \cup S_2 \cup \ldots \cup S_k$

Gesucht ist eine Möglichkeit $S$ und die Partitionierung so zu speichern, dass
die Funktionen $UNION(S_i, S_j)$ und $FIND(s)$ effizient unterstützt werden:

$UNION(S_i, S_j)$: Ändere die Partitionierung, sodass $S_i$ und $S_j$ durch
$S_i \cup S_j$ ersetzt werden

$FIND(s)$: Finde die Menge aller Teilmengen von $S$, die $s \in S$ enthalten.

Anwendungen:
\begin{itemize}
\item Finden von Zusammenhangskomponenten in Graphen
\item Algorithmus von Kruskal
\item Segmentierung eines Bildes
\item Perkolation (?)
\item Äquivalenz in FORTRAN
\end{itemize}

Realisierung als Datenstruktur

Jedes Element entspricht einem Knoten/Record/Object
Jede Menge $S_i$ wird durch einen Repräsentanten $s \in S_i$ dargestellt.
Jedes Element $s$ hat einen Verweis auf einen nachfolger $s'$, der in der selben
Menge liegt, wie $s'$. Die Verweise führen zum Repräsentanten der jeweiligen
Menge, dessen Nachfolger $\perp$ ist.
Diese Darstellung heißt disjunkter Mengen-Wald. Jeder Baum entspricht einer
Menge der Partition. Jede Wurzel entspricht dem Repräsentanten der Menge.

Implementierung von UNION und FIND

FIND(s): Eingabe: Das Objekt, das das Element im Wald darstellt
Ausgabe: Wurzel des Baumes
Algorithmus: 


\subsection{Quake-Heaps}


\section{Hashing}

\subsection{universelles Hashing}
\subsection{perfektes Hashing}
\subsection{konsistentes Hashing}
\subsection{Anwendung Bloom-Filter}


\section{Graphen}

\subsection{PageRank}

Der PageRank-Algorithmus\index{Algorithmen!Graphen!PageRank} ist ein
Algorithmus, der das Problem von Suchmaschinen lösen soll, aus einer großen Menge von Seiten, die einen bestimmten Suchbegriff
enthalten, die besten auszuwählen. Dazu gab es bereits verschiedene Ansätze,
bevor 1997 Lawrence (Larry) Page\index{Personen!Page, Lawrence} diesen
Algorithmus zum Patent anmeldete.

Er verwendet eine Idee von Jon Kleinberg\index{Personen!Kleinberg, Jon} und
Larry Page und stützt sich auf die Verweisstruktur des WWW.
Er simuliert zufällige Surfer und ermittelt die Wahrscheinlichkeit dafür, dass ein solcher eine bestimmte Seite erreicht. Seiten
mit hoher Wahrscheinlichkeit werden dann besser bewertet als solche, mit einer
geringen Wahrscheinlichkeit.

\subsubsection{Formalisierung}

Das WWW wird dargestellt als ein gerichteter Graph G = (V, E).
Dieser ist nicht zwangsläufig zusammenhängend und veränderlich. Er kann Kreise
enthalten, sowie fehlerhafte Kanten ohne Endknoten.

\subsection{Max-Fluss-Min-Schnitt}

\subsubsection{Ford-Fulkerson}
\index{Algorithmen!Graphen!Ford-Fulkerson}
Der Algorithmus von Lewster Randolph Ford jr.\index{Personen!Ford, Lewster
Randolph} und Delbert Ray Fulkerson\index{Personen!Fulkerson, Delbert Ray}

\subsubsection{Edmonds-Karp}
\index{Algorithmen!Graphen!Edmons-Karp}
Der Algorithmus von Jack Edmonds\index{Personen!Edmonds, Jack} und Richard
M. Karp\index{Personen!Karp, Richard Manning}


\section{Lineares Programmieren}

\subsection{Simplex}
\subsection{Dualität von Linearen Programmen}


\section{Komplexitätstheorie}

Das Ziel der Komplexitätstheorie ist es, Probleme nach den Ressourcen zu
klassifizieren, die für ihre Lösung notwendig sind.
Zu diesen Ressourcen zählt man insbesondere die Zeit und den Speicherplatz, aber
auch die Anzahl benötigter Zufallsbits bei randomisierten Algorithmen oder den
Kommunikationsaufwand bei verteilten Algorithmen.

Im Laufe der Zeit wurden viele verschiedene und teilweise auch äquivalente
Komplexitätsklassen definiert. Die Sammlung dieser Klassen wird manchmal auch
als Klassenzoo bezeichnet. Einige der wichtigsten Komplexitätsklassen werden
später im gleichnamigen Abschnitt vorgestellt. Genauer ansehen wollen wir uns
nun zuerst die Klassen P und NP.

\subsection{(Polynomialzeit-)Reduktion}
\subsection{verschiedene Polynomialzeitreduktionen}
\subsubsection{CLIQUE $\leq_p$ SUBSET-SUM}

\subsection{Der Klassenzoo}

\subsubsection{LOGSPACE bzw. L}
\subsubsection{PSPACE}
\subsubsection{CONP}


\part{Allgemeine Berufsvorbereitung}
\part{Allgemeine Berufsvorbereitung}
\chapter{Anwendungssysteme}
%
%
%
\section{Meta-Ebene}
\subsection{Benutzbarkeit}

Brauchbarkeit / Benutzbarkeit
	Erlernbarkeit
	Verständlichkeit
	Bedienbarkeit
		Übersichtlichkeit / Intuitivität
		Konsistenz
		Fehlerunanfälligkeit
		Bequemlichkeit / Geschwindigkeit
%
%
%
\section{Gesamtgesellschaftliche Wirkungen}
%
%
%
\section{Sicherheit}

\subsection{Fehlerbaumanalyse}

Vorgehen:
Fehler-Fall als oberstes Ereignis (TOP-Ereignis)
Verursachende Ereignisse bestimmen, dabei
	grundlegende Ereignisse (basic events) verwenden
	Einteilung von Ereignissen in korrigierbare und nicht-korrigierbare.
	Verknüpfung von Ereignissen durch logische Operatoren:
		AND, OR, XOR, Bedingte Verknüpfungen?
	paarweise stochastische Unabhängigkeit beachten

\begin{tabular}{cll}
AND	& Und-Gatter (and-gate)
	& alle Ursachen müssen vorliegen\\
OR	& Oder-Gatter (or-gate)
	& mindestens eine Ursache muss vorliegen\\
XOR	& Entweder-Oder-Gatter
	& Genau eine Ursache muss vorliegen\\
BLOCK	& Block-Gatter (inhibit-gate)
	& Nebenbedingung und die Ursache müssen vorliegen\\
INTEVT	& Zwischenereignis (intermediate event)
	& besteht aus einer Kombination von Unterereignissen\\
BSCEVT	& Basis-Ereignis (basic event)
	& wird nicht weiter untersucht\\
UDVEVT	& nicht untersuchtes-Ereignis (undeveloped event)
	& notwendige Informationen sind (noch) nicht vorhanden\\
TRANS	& Transfer-Symbol (transfer symbol)
	& verbindet Fehlerbäume
\end{tabular}
Quelle: http://opus.bibliothek.uni-augsburg.de/volltexte/2004/38/pdf/eVeroeffentlichung.pdf
%
%
%
\section{Privatsphäre}

\subsection{Definitionen und Beispiele}

\subsection{Gesetzliche Regelungen}
Wie in den Folien auch hier der Hinweis,
dass die Angaben stark vereinfacht sind und keinen Anspruch auf Vollständigkeit haben.
Die rechtsverbindliche Fassung des Gesetzes ist unbedingt dem jeweils aktuellen Gesetzestext zu entnehmen.
\subsubsection{Grundsätze}
\subsubsection{Datenschutzgesetz}
\paragraph{BDSG §1: Zweck des Gesetzes}
\paragraph{BDSG §3: Begriffsbestimmungen}
\paragraph{BDSG §3a: Datenvermeidung und Datensparsamkeit}
\paragraph{BDSG §4: Zulässigkeit der Erhebung und Nutzung}
\paragraph{BDSG §6: Unabdingbare Rechte des Betroffenen}
\paragraph{BDSG §13: Erhebung}
\paragraph{BDSG §14: Speicherung und Nutzung}
\paragraph{BDSG §19: Auskunft an den Betroffenen}
\paragraph{BDSG §20: Berichtigung und Löschung}
\paragraph{BDSG §27: Anwendungsbereich}
\paragraph{BDSG §28: Erhebung und Speicherung für eigene Zwecke}
\paragraph{BDSG §29: Geschäftsmäßige Datenerhebung und -speicherung zum Zweck der Übermittlung}
\paragraph{BDSG §30: Geschäftsmäßige Datenerhebung und -speicherung zur Übermittlung in anonymisierter Form}
\paragraph{BDSG §31: Besondere Zweckbindung}
\paragraph{BDSG §32: Beschäftigungsverhältnis}
\paragraph{BDSG §33, §42a: Benachrichtigung des Betroffenen}
\paragraph{BDSG §34: Auskunft an den Betroffenen}
\paragraph{BDSG §35: Berichtigung, Löschung und Sperrung von Daten}
\paragraph{BDSG §43, §44: Bußgeld- und Strafvorschriften}
%
%
%
\section{Arbeitswelt}
\subsection{Entschiedungsprozesse}
\paragraph{Entscheidungsprozess} Eine Abfolge von Ereignissen, die mit der Wahrnehmung eines Problemes und dem Entschluss dafür eine Lösung zu finden beginnt. Die einzelnen Ereignisse sind sowohl explizite Entscheidungen, die die beteiligten Personen treffen, als auch implizite Entschidungen, die aufgrund von Sachzwängen und dem situativen Kontext getroffen werden.
\paragraph{Mikropolitik}
\paragraph{(Macht-)spiele} Routine- und Innovationsspiele
\paragraph{Changemanagement} radikale, schnelle Veränderung (Reform) im Gegensatz zu Organisationsentwicklung im Sinne von Evolution
%
%
%
\section{Aus dem Tutorium}
\subsection{Struktogramme}
Erstellung eines Struktogrammes
\begin{enumerate}
\item Begriffe nach Grad der Spezialisierung ordnen
\item Mit dem allgemeinsten Begriff beginnen
\item Übrige Begriffe hinzufügen und Beziehungen zu anderen Begriffen kennzeichnen
\item Verwandte / ähnliche Begriffe sollten dabei in räumlicher Nähe zueinander stehen.
\end{enumerate}

\chapter{Softwaretechnik}
%
%
%
\section{Einführung}
Software; Softwaretechnik (SWT); Aufgaben der SWT; Beteiligte;
Gütemaßstab: Kosten/Nutzen; Qualität; Produkt und Prozess; Prinzip, Methode, Verfahren, Werkzeug;
technische vs. menschliche Aspekte; Arten von SWT-Situationen; Lernziele; Lernstil
"Merke"-Hinweise zu: Domänen, nichtfunktionalen Anforderungen, Kooperationsbedarf,
Projektrisiko.

Die Welt der Softwaretechnik
Routine und Innovation: Normales und radikales Vorgehen; Taxonomie: Probleme und Lösungen 

\section{Modellierung und UML}
Modelle und Modellierung (Realität vs. Modell; Phänomene vs. Konzepte); UML;

Klassendiagramme
Klassen, Attribute, Methoden
Vererbung, Aggregation, Komposition

Objektdiagramme

Sequenzdiagramme

Zustandsdiagramme (statechart)

Aktivitätsdiagramme

Anwendungsfalldiagramme

Komponentendiagramme

Kollaborationsdiagramme,
Inbetriebnahmediagramme,
Kommunikationsdiagramme,
Interaktions-Übersichts-Diagramme

UML Metamodell; Profile;
Notationsdetails
(Klassen, Assoziationen, Schnittstellen, Zustände) 

\section{Ermitteln WAS}

\subsection{Anforderungsbestimmung}
Erhebung (Requirements Elicitation): Anforderungen und Anforderungsbestimmung
(Requirements Engineering); Arten von Anforderungen; Anforderungen und Modellierung;
Harte und weiche Systeme; Probleme und Chancen erkennen;
Erhebungstechniken (herkömmliche, darstellungs-basierte, soziale, wissenserhebende) 

\subsubsection{Anwendungsfälle (Use Cases)}

Was ist ein Use Case?;
Wichtige Parameter (Bereich, Detailgrad/Zielniveau);
schrittweise Präzisierung;
Use-Case-Hierarchien (Überblick, Benutzerziele, Details);
SuD System under discussion

Checkliste für Use Cases
\begin{description}
\item[Titel]
\begin{itemize}
	\item Aktive Verbalphrase, die das Ziel des Hauptakteurs nennt?
	\item Ist das SuD für die Zielerreichung "zuständig"?
	\end{itemize}
\item[Bereich (SuD)]
\begin{itemize}
	\item Angegeben?
	\item Wenn das SuD entworfen werden muss, müssen (a) alle Teile entworfen werden und (b) nichts außerhalb (Systemgrenze)?
	\end{itemize}
\item[Detailgrad/Zielniveau]
\begin{itemize}
	\item Liegt das Ziel wirklich auf diesem Niveau?
	\item Passt der Inhalt zum geplanten Detailgrad?
	\end{itemize}
\item[Hauptakteur]
\begin{itemize}
	\item Hat er/sie/es Verhalten?
	\item Hat er/sie/es ein Ziel, das zu einer Dienstzusicherung des SuD passt?
	\end{itemize}
\item[Voraussetzungen]
\begin{itemize}
	\item Sind sie verbindlich und herstellbar?
	\item Werden sie im Use Case nicht mehr überprüft?
	\end{itemize}
\item[Beteiligte und ihre Interessen]
	Muss das SuD in diesem Use Case diese Interessen bedienen?
\item[Mindestzusicherungen]
	Sind die Interessen aller Beteiligten angemessen geschützt?
\item[Zusicherungen im Erfolgsfall]
	Sind die Interessen aller Beteiligten befriedigt?
\item[Haupt-Erfolgsszenario]
\begin{itemize}
	\item Hat es 3 bis 9 Schritte?
	\item Beschreibt es den Ablauf vom Auslöser bis zur Erfolgsgarantie?
	\item Erlaubt es ggf. geeignete Abwandlungen in der Reihenfolge?
	\end{itemize}
\item[Jeder Einzelschritt]
\begin{itemize}
	\item Ist er als erreichtes Ziel formuliert?
	\item z.B. "validieren", nicht: "prüfen"
	\item Treibt er den Prozess sichtbar voran?
	\item Ist klar, wer der handelnde Akteur ist?
	\item Ist die Absicht des Akteurs klar?
	\item Abstrahiert der Schritt von der Bedienschnittstelle?
	\item Ist erkennbar, welche Information verarbeitet wird?
	\end{itemize}
\item[Erweiterungsbedingungen]
\begin{itemize}
	\item Kann das SuD diesen Fall entdecken (falls nötig)?
	\item Muss das SuD diesen Fall behandeln?
	\end{itemize}
\item[Inhalt des Use Cases insgesamt]
\begin{itemize}
	\item Gegenüber den Beteiligten: "Ist dies, was Du möchtest?",
	"Kannst Du später entscheiden, ob es richtig gebaut wurde?"
	\item Gegenüber den Entwicklern: "Kannst Du das bauen?"
	\end{itemize}
\end{description}

Anwendungsfalldiagramme

\section{Verstehen WAS}

\subsection{Analyse (statisches Objektmodell)}
Von Use-Cases zu Klassen,

Abbott's Methode (Substantive sind Kandidaten für Klassen, Verben für Operationen,
Adjektive für Attribute, Eigennamen für Objekte, "ist ein" für Vererbung etc.);

Checklisten zur Identifikation von Klassen, Assoziationen, Attributen, Operationen, Vererbung;
Entwicklerrollen und Modellarten (Analysemodell vs. Entwurfsmodell) 

\subsection{Analyse (dynamisches Objektmodell)}
Klassen finden mit dynamischer Modellierung; Zustandsdiagramme (statechart diagrams);
Sequenzdiagramme; Aufbau eines Anforderungsanalyse-Dokuments;
Validierung (und Gegensatz zu Verifikation) 

\section{Entscheiden WIE}

\subsection{Software-Architektur}
Architektur=Gesamtstruktur; Erfüllen nichtfunktionaler und funktionaler Anforderungen;
globale Eigenschaften; wiederverwendbare Architekturen (Standard-Architekturen);
Architekturstile (zum Selbstentwickeln von Architekturen);
Modularisierung (Modulbegriff, Aufteilungskriterien)

Es gibt für eine Reihe wiederkehrender Mengen
nichtfunktionaler Anforderungen etablierte SW-Architekturen
oder zumindest Architekturstile
\begin{itemize}
\item Klient-/Dienstgeber-Architektur (client/server arch.)
\item eine einfache Sorte verteilter Architekturen
\item Ereignisgesteuertes System (event-based arch.)
\item eine Sorte lose gekoppelter Architekturen
\item Ablage-basierte Architektur (repository arch.)
\item noch eine Sorte lose gek. A.; oft mit Ereignissteuerung verbunden
\item Unterbrechungsorientiertes System (interrupt-based system)
\item eine Architektur für kleinere Echtzeitsysteme
\item Mehrschicht-Architektur (layered arch.)
\item ein allgemeiner A.stil, der mit vielen anderen A.ideen verbunden werden kann
\item JavaEE-Architektur, CORBA-Architektur, .NET-Architektur
\item technologiezentrierte Architekturen
\end{itemize}

\subsection{Modularisierung}
Modulbegriff; Kriterien für Aufteilung;
Fallstudie: KWIC;
KWIC 1: Datenflusskette; Einschätzen der Entwurfsqualität;
KWIC 2: Zentrale Steuerung;
KWIC 3: Datenabstraktion; Verhalten bei Änderungen; Verwandtschaft mit Architekturstilen 

\subsection{Entwurfsmuster}
Was macht ein Problem schwierig?; Einfachheit durch Wiedererkennen von Mustern;
Idee von Entwurfsmustern;
Kompositum-Muster (composite pattern);
Adapter-Muster (adapter pattern);
Brücken-Muster (bridge pattern);
Fassaden-Muster (facade pattern) 

Arten von Entwurfsmustern;
Stellvertreter-Muster (proxy pattern);
Kommando-Muster (command pattern);
Beobachter-Muster (observer pattern);
Strategie-Muster (strategy pattern);
Abstrakte-Fabrik-Muster (abstract factory pattern);
Erbauer-Muster (builder pattern) 

\subsection{Schnittstellenspezifikation}
Sichtbarkeiten (public, protected, private, package),
Spezifikation von Voraussetzungen (preconditions) und Wirkungen (postconditions) mit OCL
(context, pre, post, inv);
Abbildung von Assoziationen in Code

\subsection{Qualitätssicherung}

Analytische Qualitätssicherung
Defekttest;
Auswahl der Eingaben (Funktionstest, Strukturtest); Auswahl der Testgegenstände (bottom-up, top-down, opportunistisch);
Ermittlung des erwarteten Verhaltens (Referenzsystem, (Teil)Orakel); Wiederholung von Tests (Rückfalltesten, Testautomatisierung) 
Testautomatisierung (Werkzeuge, Strukturierung, JUnit);
Stoppkriterien für das Testen;
Defektortung;
Benutzbarkeitstest;
Lasttest;
Akzeptanztest;
manuelle statische Prüfung (Durchsicht; Inspektion; Perspektiven-basiertes Lesen);
automatische statische Prüfung (Modellprüfung; Quelltextanalyse) 

Konstruktive Qualitätssicherung (Qualitätsmgmt., Prozessmgmt.)
Projekt- vs. Prozessmgmt.;
Arten von Prozessmgmt.-Leitlinien;
CMM-SW/CMMI (5 Prozessreifestufen);
Total Quality Management (TQM) (Prinzip: Kundenzufriedenheit); ISO 9000 

\subsection{Prozessmodelle}

Bestandteile eines Prozessmodelles:

Rollen
\begin{itemize}
\item Welche gibt es und wie genau sind sie definiert ("Arbeitsbeschreibungen")?
\end{itemize}

Aktivitäten
\begin{itemize}
\item Welche sind vorgesehen? Wie genau sind sie definiert?
\item Entscheidungsspielraum über Einsatz oder Art des Einsatzes?
\item Gibt es Checklisten? Vorgaben über Werkzeuge, Methoden, Richtlinien?
\end{itemize}

Artefakte
\begin{itemize}
\item Welche sind vorgesehen? Wie genau sind sie definiert? Wie verbindlich ist das?
\item Gibt es Vorlagen (Schablonen)?
\end{itemize}

Steuerung der Aktivitäten
\begin{itemize}
\item Gibt es einen festen Ablaufplan? Oder weichere Kriterien für die Abfolge von Aktivitäten?
\item Sind Eintritts- und Austrittsbedingungen definiert?
\end{itemize}

Prozessmodell-Auswahlkriterien

\subsubsection{Wasserfallmodell}

Das Wasserfallmodell ist ein lineares (nichtiteratives) Vorgehensmodell in der Softwareentwicklung, bei dem der Softwareentwicklungsprozess in Phasen organisiert wird.
Dabei gehen die Phasenergebnisse wie bei einem Wasserfall immer als bindende Vorgaben für die nächsttiefere Phase ein.

Im Wasserfallmodell hat jede Phase vordefinierte Start- und Endpunkte mit eindeutig definierten Ergebnissen.
In Meilensteinsitzungen am jeweiligen Phasenende werden die Ergebnisdokumente verabschiedet.
Zu den wichtigsten Dokumenten zählen dabei das Lastenheft sowie das Pflichtenheft.

\begin{enumerate}
\item Planung
\item Anforderungsbestimmung
\item Architekturentwurf
\item Feinentwurf
\item Implementierung
\item Integration
\item Validierung
\item Inbetriebnahme
\end{enumerate}

\subsubsection{Prototypmodell}

\subsubsection{Evolutionäre Modelle}

\subsubsection{Spiralmodell}

\subsubsection{Flexiblere Planung (Agile Methoden)}

\subsubsection{Anpassbare Prozessmodelle}
Rational Unified Process (RUP)
V-Modell XT;
Erklärung "Agile Methode" (XP) 


\subsection{Projektmanagement}
Was und wofür?; Aufgabenfelder;
Schätzen (Schätzverfahren; Funktionspunktschätzung); Todesmarschprojekte 
Zeit- und Ressourcenplanung;
Microsoft Project;
Critical Path Method (CPM);
Finden von Aufgabenzerlegungen;
Risikomanagement; Risikolisten; DOs and DON'Ts 

Teams; Sportteam oder Chor?; Organisationsstrukturen; Rollen; Kommunikationsstrukturen;
psychologische Faktoren; Schätzen von Wahrscheinlichkeiten;
Motivation; Attribution; Haltungen; soziale Einflüsse 
Projektplan; Projektleitung;

nichtlineare Dynamik (Brook's Gesetz; Selbstverstärkung von Qualitätsmängeln;
Teufelskreis von Qualität und Zeitdruck);
Kommunikation (geplant/ungeplant, synchron/asynchron); Medien; Besprechungen 

Normales Vorgehen maximieren: Wiederverwendung
Arten der Wiederverwendung (Produkt/Prozess; Gegenstand; Ziel);
Risiken und Abwägung; Hindernisse; Produktivität; Wiederverwendung für normales Vorgehen;
Muster; Arten von Mustern;
Prinzipien (Abstraktion, Strukturierung, Hierarchisierung, Modularisierung, Lokalität, Konsistenz, Angemessenheit, Wiederverwendung, Notationen);
Analysemuster 
Benutzbarkeitsmuster; Prozessmuster; Mustersprachen; Anti-Muster; Werkzeuge 

Wissen weitergeben:
Dokumentation
Arten von Dokumentation; Qualitätseigenschaften (übersichtlich, präzise, korrekt, hilfreich);
positive und negative Beispiele; Prinzipien (Selbstdokumentation, Minimaldokumentation);
Begründungsmanagement (Fragen + Vorschläge + Kriterien + Argumente ergeben Entscheidungen) 

\chapter{Arbeits- und Lebensmethodik}
%
%
%
\section{Einführung}
%
%
%
\section{Ziele oder Wünsche, Handeln oder Nur-Überlegen, Lösungen oder Probleme}
%
%
%
\section{Subjektivität oder Objektivität, Freude oder Unaufmerksamkeit, Freiwilligkeit oder Zwang, Dankbarkeit oder Ärger}
%
%
%
\section{Vertrauen oder Angst}
%
%
%
\section{Konzentration oder Zeitverschwendung, Entspannung oder Stress}
%
%
%
\section{Entscheiden oder Zagen, Verantwortung oder Fremdbestimmung}
%
%
%
\section{Programme entdecken und bearbeiten}
%
%
%
\section{Klarheit oder Überraschung, Zuwendung oder Gleichgültigkeit}
%
%
%
\section{Motivation oder Unlust, Akzeptieren oder Sich-Abfinden}
%
%
%
\section{Werte oder Beliebigkeit, Konsequenz oder Scheitern}

\subsection{Begriffe}

\paragraph{Wert} Eine Idee, die mir dauerhaft wichtig ist.
Erkennbar daran, dass ich konsequent danach handele.

\paragraph{Konsequenz} besteht darin, Werte im Handeln nicht zu verletzen und nicht dauerhaft zu vernachlässigen.

\paragraph{Lebensaufgabe} Ein zeitlich und inhaltlich unbegrenzter Wunsch (also kein Ziel), dem ich mein Leben lang als Leitlinie für mein Verhalten folge.

\subsection{Werte}

Ablenkung
Anderssein
Anerkennung
Angstfreiheit
Anstand
Attraktivität
Ausgeglichenheit
Autarkie
Autonomie
Balance
Benehmen
Beschäftigung
Chancengleichheit
Chaos
Demokratie
Disziplin
Ehrgeiz
Einfluss
Eleganz
Engagement
Erfolg
Erwartung
Fairness
Familie
Fitness
Fleiß
Flexibilität
Freiheit
Freizeit
Freude
Führung
Fürsorge
Geborgenheit
Gelassenheit
gerechtigkeit
Gesundheit
Gleichberechtigung
Gleichheit
Glück
Härte
Heimat
Hilfsbereitschaft
Hoffnung
Individualität
Kinder
Kooperation
Kompromiss
Konfliktfähigkeit
Konsequenz
Kontrolle
Kraft
Kultur
Leben
Liebe
Macht
Nachhaltigkeit
Naturschutz
Nächstenliebe
Offenheit
Ordnung
Privatsphäre
Pünktlichkeit
Qualität
Reichtum
Respekt
Ruhe
Schmerz
Schönheit
Selbstbestimmung
Sicherheit
Sitte
Solidarität
Sorge
Spaß
Sport
Streben, Strebsamkeit, Zielstrebigkeit
Tierschutz
Toleranz
Tradition
Trauer
Überleben
Überlegenheit
Umweltschutz
Unauffälligkeit
Verantwortung
Vergnügen
Verlangen
Verlässlichkeit
Vernunft
Vertrauen
Vorbildsein
Vorfreude
Wahrheit
Wehmut
Wertschätzung
Wissen
Wohlfühlen
Zeitvertreib
Zuverlässigkeit
Zweckmäßigkeit

\subsection{Lebensaufgabe}
Ein Ansatz: Wenn ich Nichts mehr tun muss, aber alles tun darf und kann, was tue ich?
Dinge / Wünsche, die einen motivieren / ansprechen, aber nicht zeitlich befristet sind (also nicht zur Zielsetzung geeignet), sind Kandidaten für Werte oder eine Lebensaufgabe.

\subsection*{Gewohnheiten verschieben}
Ich üernehme immer mehr Verantwortung für mein eigenes Leben
Ich weiß, was ich will.
\subsection*{Hausaufgabe}
Ich finde die drei wichtigsten Beispiele für Konsequenz oder Mangel daran aus meinem Leben.
Was war jeweils die Folge daraus?
Ich formulliere eine Hypothese, was meine Lebensaufgabe sein könnte.

%
%
%
\section{Alles oder Nichts}

\subsection*{Hausaufgabe}
Ich ergänze oder ändere meine Zielsetzung so, dass mein notleidenster Lebensbereich deutlich mehr Geltung bekommt.
%
%
%
\section{Soft-Skills}

Zeitmanagement
Mediation, Konfliktmanagement
Moderation, Gesprächsführung
Teamarbeit
Rhetorik, Präsentation

\subsection{Multiple Intelligenzen}
\begin{itemize}
\item logisch-mathematische
\item sprachlich-linguistische
\item bildlich-räumliche
\item körperlich-kinästhetische
\item musikalisch-rhythmische
\item interpersonelle
\item intrapersonelle
\end{itemize}

\subsection{Kompetenzerwerb}
unbewusste Inkompetenz
bewusste Inkompetenz
bewusste Konpetenz
unbewusste Kompetenz

\subsection*{Gewohnheiten verschieben}
Ich lerne hinzu, indem ich in der Praxis übe und Könner frage bzw. ein Buch lese.
Ich pflege auf jedem Sektor einen Stil, der zu mir passt.
Ich werde vor allem meine Stärken stärken, anstatt gegen meine Schwächen anzugehen.



\part{Notizen}

AlP1
	Haskell
	Lambda-Kalkül
	SKI-Logik
	Register-Maschine
	Primitive Rekursion
	Turing-Maschinen
AlP2
	OOP
	JAVA
	Python
AlP3
	Datenstrukturen
	Graphen
	Algorithmen
AlP4
AlP5
MafI1
	Logik und diskrete Mathematik
MafI2
	Analysis
MafI3
TI1, TI2
TI3
TI4
GTI
	Formale Sprachen
	Automaten, Sprachklassen, Grammatiken
	DEA, NEA
	Turing-Maschinen

\bibliography{src/parts/references.bib}{}
\bibliographystyle{plain}

\end{document}
