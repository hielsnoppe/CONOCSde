%
%
%
%
\section{Motivation}

Definition: Algorithmus\\
Ein Algorithmus ist eine Rechenvorschrift, die eine Eingabe in eine Ausgabe überführt, und:
\begin{enumerate}
	\item endlich beschrieben,
	\item effektiv,
	\item allgemein,
	\item (deterministisch)
\end{enumerate}
ist.
Ein Algorithmus ist gut, wenn er:
\begin{enumerate}
	\item immer korrekt terminiert.
	\item richtige Antworten gibt.
	\item schnell,
	\item speichereffizient,
	\item gut lesbar
\end{enumerate}
ist.

Bewertung nach Laufzeit und Speicherbedarf, experimentell oder theoretisch\\
Analyse-Verfahren nach den Datenstrukturen und Algorithmen\\
Zum Schluss Anwendungen

\subsection{Bewertung und Analyse von Algorithmen}

Analyse auf Basis von Pseudo-Code, eine Zeile PC entspricht beliebig aber konstant vielen Schritten auf RAM.
%
%
\subsubsection{Registermaschinen und O-Notation}
Empirische Analyse vs. theoretische Analyse\\
Definition: Registermaschine bzw. Random Access Machine (RAM)\\
Definitionen: Laufzeit und Speicherbedarf bei einer RAM
%
%
\subsubsection{Amortisierte Analyse}
Wir betrachten eine Folge von Operationen eines Algorithmus im worst-case.
Wenn wir zeigen können, dass jede Folge von $n$ Operationen $f(m)$ Zeit benötigt,
dann sagen wir, die amortisierten Kosten pro Operation sind $f(m)/n$.
%
%
\subsubsection{Analyse randomisierter Algorithmen}
