\section{Wörterbücher}

Der abstrakte Datentyp Wörterbuch (engl. \textit{dictionary}) ermöglicht es, Tupel zu speichern,
die aus Schlüsseln $k$ (engl. \textit{keys}) aus einer Schlüsselmenge $K$ bestehen,
sowie aus Werten $v$ (engl. \textit{values}) aus einer Wertemenge $V$,
auf die die Schlüssel abgebildet werden.

Je nachdem, ob die Schlüsselmenge geordnet ist oder nicht,
spricht man von einem ungeordneten oder einem geordneten Wörterbuch.

%
%
\subsection{ungeordnetes Wörterbuch}

\subsubsection{Operationen}
\paragraph{put $k$ $v$} Fügt einen Wert $v$ mit $k$ als Schlüssel in das Wörterbuch ein.
\paragraph{get $k$} Gibt den Wert zu einem Schlüssel $k$ aus dem Wörterbuch zurück.
\paragraph{remove $k$} Entfernt den Schlüssel $k$ und den dazugehörigen Wert aus dem Wörterbuch.

\subsubsection{Implementierung}

Implementierung durch Hashing (\ref{subsec:Hashing}) mittels Arrays...

%
%
\subsection{geordnetes Wörterbuch}

Bei einem geordneten Wörterbuch geht man von einer total geordnete Schlüsselmenge $K$ aus.
Das heißt, für alle $k_1, k_2, k_1 \neq k_2 \in K$ gilt entweder $k_1 < k_2$ oder $k_2 < k_1$.

\subsubsection{Operationen}
Durch diese Eigenschaft können ergeben sich zusätzlich zu den Operationen eines ungeordneten Wörterbuches folgende Operationen:

\paragraph{min} Gibt den kleinsten Schlüssel des Wörterbuches zurück.
\paragraph{max} Gibt den größten Schlüssel des Wörterbuches zurück.
\paragraph{succ $k$} Gibt den nächstgrößeren Schlüssel als $k$ (Nachfolger, engl. \textit{successor}) aus dem Wörterbuch zurück.
\paragraph{pred $k$} Gibt den nächstkleineren Schlüssel als $k$ (Vorgänger, engl. \textit{predecessor}) aus dem Wörterbuch zurück.

\subsubsection{Implementierung}

Eine einfache Möglichkeit zur Implementierung ist ähnlich wie bei einem ungeordneten Wörterbuch vorzugehen und jeweils Suchoperationen durchzuführen.
Es ist aber leicht zu erkennen, dass dies bei größeren Datenmengen sehr ineffizient ist.

Viele effizientere Möglichkeiten, geordnete Wörterbücher zu implementieren,
lernen wir im folgenden Abschnitt über Bäume (\ref{sec:Baeume}) kennen.
