\section{Bäume}
\label{sec:Baeume}

Zu jeder der Datenstrukturen werden im Folgenden alle Operationen eines geordneten Wörterbuches für Elemente $e$ definiert,
die in einer konkreten Implementierung eines Wörterbuches die Schlüssel darstellen.
Zusätzlich müssen dann Verweise auf die jeweils zu einem Schlüssel gehörigen Werte gespeichert werden.
Wie das im Einzelnen realisiert werden kann, ist an den jeweiligen Stellen erläutert.

\begin{center}
\begin{tabular}{|p{.4\linewidth}|p{.5\linewidth}|}
\hline
Datenstruktur		& Laufzeiten	\\\hline\hline
\paragraph{Binäre Suchbäume}
&
\\\hline
\paragraph{AVL Bäume}
&
$$\mathcal{O}(\log n)$$
\\\hline
\paragraph{Rot-Schwarz-Bäume}
&
\\\hline
\paragraph{$(a, b)$-Bäume}
&
\\\hline
\paragraph{Heaps}
&
$$\mathcal{O}(\log n)$$
\\\hline
\paragraph{Tries}
&
\\\hline
\paragraph{Skip-Listen}
&
\\\hline
\end{tabular}
\end{center}

\subsection{Binäre Suchbäume}

Operationen jeweils einmal rekursiv und einmal iterativ
\paragraph{Einfügen}
\paragraph{Suchen}
\paragraph{Löschen}

\subsection{AVL-Bäume}
AVL-Bäume sind binäre Suchbäume, die einem weiteren Kriterium genügen.
Sie vermeiden es nämlich, dass der Baum "`entartet"', also besonders stark ungleichmäßig wächst.
Dadurch verbessert sich die Laufzeit der Operationen auf ihnen.
Die Operationen \textit{Einfügen}, \textit{Suchen} und \textit{Löschen} werden ebenso durchgeführt,
wie in einfachen binären Suchbäumen.
Um jedoch eine Entartung zu vermeiden,
wird ein entstehendes Ungleichgewicht nach dem \textit{Einfügen} oder \textit{Löschen} durch Rotationen ausgegleichen.
Dabei bleibt die binäre-Suchbaum-Eigenschaft erhalten.

\paragraph{Rotation, Doppel-Rotation}

\subsection{Rot-Schwarz-Bäume}
Ein anderes Konzept um die Entartung von binären Suchbäumen zu vermeiden, sind Rot-Schwarz-Bäume.
In ihnen werden jedem Knoten eine Farbe, rot oder schwarz,
zugewiesen und darüber erweiterte Anforderungen an die Struktur des Baumes gestellt:
\begin{itemize}
	\item Die Wurzel des Baumes ist immer schwarz.
	\item Die $\bot$-Knoten sind immer schwarz.
	\item Die Nachfolger roter Knoten sind schwarz.
	\item Auf dem Pfad von der Wurzel zu jedem $\bot$-Knoten liegen immer gleich viele schwarze Knoten.
\end{itemize}

Java-Applet unter http://fbim.fh-regensburg.de/~saj39122/gikasch/start.html

\paragraph{Suchen} Die Suche in einem Rot-Schwarz-Baum verläuft wie in jedem binären Suchbaum.
Ihre asymptotische Laufzeit liegt in $\mathcal{O}(...)$.
\paragraph{Einfügen}
\paragraph{Löschen}

\subsection{$(a, b)$-Bäume}
Anders als die vorigen Baumstrukturen sind $(a, b)$-Bäume keine binäre Stuktur und wachsen nicht von der Wurzel in Richtung ihrer Blätter, sondern umgekehrt.
Anstelle von zwei möglichen Kindern haben die Knoten eines $(a, b)$-Baumes zwischen $a$ und $b$ (jeweils einschließlich, $a, b \in \mathbb{N}$) viele Kinder.
Eine Ausnahme ist die Wurzel, die 2 bis $b$ viele Kinder hat.
Weiterhin enthalten die Knoten selbst zwischen $a-1$ und $b-1$ (s.o.) viele Schlüssel.
Die Suchbaumeigenschaft findet sich darin wieder,
dass die Schlüssel in sortierter Reichenfolge gespeichert werden und die Verweise auf die Kinder zwischen den einzelnen Schlüsseln angelegt sind.
Alle Schlüssel eines Kindes sind also größer als der Schlüssel links und kleiner als der Schlüssel rechts von dem Verweis auf das Kind.

\paragraph{Suchen} Die Suche in einem $(a, b)$-Baum hat eine asymptotische Laufzeit in $\mathcal{O}(\log n)$.

\paragraph{Einfügen} Neue Werte werden in einem $(a, b)$-Baum immer in einem Blatt eingefügt.
Wenn dieses durch den neuen Schlüssel "`überläuft"',
so wird es geteilt und ein Schlüssel daraus in einen übergeordneten Knoten verschoben.
Siehe dazu die Operation \textit{Teilen}.

Das Einfügen hat eine asymptotische Laufzeit in $\mathcal{O}(...)$.

\paragraph{Löschen} Gelöscht werden Werte immer aus den Blättern.
Um einen Wert aus einem inneren Knoten zu löschen,
wird dieser zuerst mit seinem symmetrischen Vorgänger bzw. Nachfolger ersetzt.
Dieser ist immer ein Blatt, sodass er problemlos seine ursprüngliche Position verlassen kann.
In beiden Fällen ist es jedoch möglich, dass anschließend das Blatt, aus dem gelöscht wurde, zu wenige Werte enthält.
Man sagt auch, er "`läuft unter"' analog dazu, wenn ein Knoten überläuft.
Um die Struktur des Baumes dann wieder herzustellen, gibt es unterschiedliche Balancierungsoperationen:

\subparagraph{Auffüllen} (Auch Leihen oder Rotation) Dabei wird ein Wert aus dem Elternknoten in den unterlaufenen Knoten abgegeben.
Um dies wieder auszugleichen wird versucht,
einen Wert von einem benachbarten Geschwisterknoten des unterlaufenen Knotens in den Elternknoten zu verschieben.
Wenn dieser dann aber selbst nicht mehr ausreichend Werte enthält,
wird durch einen Knoten aus dem nächsthöheren Elternknoten ausgegelichen.
Dieser muss dann wiederum durch die gleichen Operationen ausbalanciert werden.

\paragraph{Teilen} Wenn ein Knoten überläuft, so muss er geteilt werden.
Dazu wird der mittlere Wert aus dem übergelaufenen Knoten in dessen Elternknoten eingefügt und alle Werte links und rechts davon in jeweils einen neuen Knoten gespeichert,
der links bzw. rechts von dem aufgestiegenen Wert in dem Elternknoten steht.

\subsection{Heaps}
\label{subsec:Heaps}

Unterscheidung Min-Heap und Max-Heap.\\
Heapbedingung: Alle Kinder eines Knotens größer (Min-Heap) bzw. kleiner (Max-Heap) als Eltern-Knoten.\\
Implementierung als (binärer) Baum verlinkt, oder mit (dynamischem) Array.\\
Als Sonderfall Treaps, die bezüglich zweier Werte einmal Heap- und einmal Binäre-Suchbaum-Eigenschaft haben.

\subsection{Tries}
\label{subsec:Tries}

Tries sind eine Möglichkeit, um Zeichenketten (engl. \textit{strings}) effizient zu verwalten.
Man kann die gespeicherten Zeichenketten auch als Schlüssel für andere Daten verwenden und so ein geordnetes Wörterbuch implementieren.

Die Struktur von Tries sind Bäume, bei denen einzelne Zeichen der Zeichenketten in den Kanten zwischen den Knoten gespeichert werden.
Die Knoten selbst enthalten keine Werte.
Man kann aber beispielsweise die Blätter dazu nutzen Verweise zu speichern, um ein Wörterbuch zu implementieren.
Weiterhin wird zusätzlich zu den Zeichenketten als letztes Zeichen jeweils ein Symbol eingefügt, das das Wortende markiert.
Hier soll dafür das \EURdig-Zeichen dienen.

\begin{figure}[h]%[htbp]
%\includegraphics{unkomprimiertertrie.png}
\caption{unkomprimierter Trie}
\label{fig:UnkomprimierterTrie}
\end{figure}

Um besonders die Suche in Tries zu beschleunigen, kann man komprimierte Tries (auch Patricia-Tries, von engl. \textit{Practical Algorithm to Retrieve Information Coded in Alphanumeric}) einsetzen.
Dabei werden mehrere Kanten, die auf einem Pfad liegen, von dem keine Verzweigungen ab gehen, zu einer zusammengefasst.

\begin{figure}[h]%[htbp]
%\includegraphics{komprimiertertrie.png}
\caption{komprimierter Trie}
\label{fig:KomprimierterTrie}
\end{figure}

Tries können auch als Suffixbaum zum Suchen in Zeichenketten verwendet werden.
Mehr dazu findet sich im Abschnitt über die Arbeit mit Zeichenketten (\ref{subsec:ArbeitenMitZeichenketten}).

\subsubsection{Operationen}

\paragraph{put $e$}
\paragraph{get $e$}
\paragraph{remove $e$}
\paragraph{min}
\paragraph{max}
\paragraph{succ $e$}
\paragraph{pred $e$}

%\subsection{Rank-balanced Trees}
%http://www.cs.princeton.edu/~sssix/papers/rb-trees.pdf

\subsection{Skip-Listen}
\label{subsec:SkipListen}

Skip-Listen sind eine sogenannte "`randomisierte Datenstruktur"'.
Das bedeutet, dass ihre Struktur zwar grundsätzlich festgelegt ist,
jedoch durch die gezielte Verwendung von zufälligen Entscheidungen das genaue Aussehen nicht eindeutig ist.
Auch wenn also mehrfach die gleichen Daten in der gleichen Reihenfolge gespeichert werden,
kann die Datenstruktur jeweils unterschiedlich aussehen.

Die Grundlage für Skip-Listen bilden mehrere einfach verkettete Listen,
die als Ebenen übereinander liegend betrachtet werden.
Auf der untersten Ebene werden Elemente wie in einer normalen sortierten verketteten Liste verwaltet.

Jedes Element kann aber auch in einer zufälligen Anzahl von Ebenen darüber redundant gespeichert sein.
Neben einem Verweis auf seinen Nachfolger innerhalb der Ebene hat es dann auch einen Verweis auf seinen Repräsentanten in der darunterliegenden Ebene.

\begin{figure}[h]%[htbp]
%\includegraphics{skiplist.png}
\caption{Skip-Liste}
\label{fig:SkipListe}
\end{figure}

Dabei verhält es sich so, dass je höher eine Ebene ist,
desto geringer die Wahrscheinlichkeit für ein bestimmtes Element ist, in dieser Ebene vorhanden zu sein.
Dadurch enthalten die höheren Ebenen weniger Elemente als die niedrigeren und können mit weniger Schritten durchlaufen werden.
Diese Eigenschaft ermöglicht es, effiziente Operationen auf der Datenstruktur zu implementieren.

Zur Implementierung eines Wörterbuches kann man zu jedem Element auf der untersten Ebene einen Verweis auf den dazugehörigen Wert speichern.

\subsubsection{Operationen}

\paragraph{put $e$}
\paragraph{get $e$}
\paragraph{remove $e$}
\paragraph{min}
\paragraph{max}
\paragraph{succ $e$}
\paragraph{pred $e$}

\subsubsection{Mathematische Grundlagen}
