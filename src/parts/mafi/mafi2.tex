\section{Aufbau des Zahlensystems I}
\subsection{Natürliche Zahlen, Peanosches Axiomensystem}
\subsection{Ganze Zahlen}
\subsection{Halbgruppe, Gruppe, Ring, Elementare Eigenschaften}
\subsection{Rationale Zahlen als Aquivalenzklassen, Körper}
\subsection{Reelle Zahl als unendliche Dezimalbrüche und als Aquivalenzklasse von Cauchy-Folgen}
\subsection{algebraische und transzendente reelle Zahlen}
\subsection{Ordnungseigenschaften reeller Zahlen, Begriffe: lineare Ordnung, Schnitt, Schranke, Min/Max, Inf/Sup}
\subsection{Betragsfunktion, Rechnen mit Ungleichungen, Bernoulli-Ungleichung, Ungleichung von Cauchy-Schwarz}

\section{Der Polynomring R[x]}
\subsection{Rechenregeln, Ringstruktur}
\subsection{Effektives Polynomauswerten mittels Horner-Schema}
\subsection{Nullstellen, Linearfaktoren, Faktorisierung reeller Polynome}
\subsection{Bestimmung rationaler Nullstellen bei rationalen Polynomen}
\subsection{Ganz rationale, echt gebrochen rationale Funktionen}
\subsection{Polynomdivision, Euklidscher Algorithmus}
\subsection{Interpolation mit Polynomen, Lagrange-Interpolation, Newton-Interpolation}
\subsection{Eine informatische Anwendungen: Karp-Rabin-Fingerprint zum String Matching}

\section{Aufbau des Zahlensystems II: Komplexe Zahlen}
\subsection{komplexe Zahlenebene}
\subsection{Rechnen mit komplexen Zahlen, Körperstruktur}
\subsection{Betragsfunktion, konjugiert komplexe Zahl}
\subsection{komplexe Polynome, Fundamentalsatz der Algebra}
\subsection{Faktorisierung reeller Polynome in lineare und quadratische Terme}
\subsection{Polarkoordinaten, komplexe Exponentialfunktion}
\subsection{Formeln von De Moivre, Rechnen in Polardarstellung, Potenzieren, Radizieren}
\subsection{Einheitswurzeln}
\subsection{Historisch: Lösung kubischer Gleichungen}
\subsection{Überlagerung von Schwingungen}

\section{Folgen und Grenzwerte}
\subsection{Begriff der Folge, Beispiele, Konvergenz/Divergenz, Eindeutigkeit des Grenzwertes}
\subsection{Konvergenzkriterien: Vergleichskriterium, Cauchy, beschränkte monotone Folgen}
\subsection{Rechenregeln für Grenzwerte}
\subsection{Folge der Partialsummen (Reihe)}
\subsection{Geometrische Folgen und Reihen, Anwendung: Koch’sche Eisblume}
\subsection{harmonische Reihe und alternierende harmonische Reihe}
\subsection{Eulersche Zahl als Grenzwert}
\subsection{O-Notation: Asymptotisches Wachstum, $\mathcal{O}(f(n))$, $\Omega(f (n))$, $\Theta(f (n))$, $o(f(n))$, $\omega(f (n))$, Rechenregeln, Wachstum von Standardfunktionen, Stirling-Formel}

\section{Stetigkeit von Funktionen, Differentiation}
\subsection{Grenzwert einer Funktion in einem Punkt, einseitige Grenzwerte, Beispiele}
\subsection{Grenzwertarithmetik für Funktionen, das Einschnüren von Funktionstermen (Bsp. $\lim \sin x/x)$}
\subsection{Asymptoten des Funktionsgraphen, Polstellen gebrochen rationaler Funktionen}
\subsection{Stetigkeit über einem Intervall, $\delta$-Kriterium}
\subsection{Stetige Fortsetzung, gleichmäßige Stetigkeit}
\subsection{Eigenschaften stetiger Funktionen: Komposition stetiger Funktionen ist stetig, Nullstellensatz, Zwischenwertsatz, Min-Max-Eigenschaft}
\subsection{Ableitung in einem Punkt, Differenzierbare Funktionen sind stetig}
\subsection{Geometrische Interpretation: Anstieg der Tangente, Analytische Interpretation:
Lineare Approximation}
\subsection{Rechenregeln zum Differenzieren, höhere Ableitungen}
\subsection{Ableitung trigonometrischer Funktionen}
\subsection{Extremalstellen einer Kurve, Mittelwertsatz und seine Verallgemeinerung}
\subsection{Bestimmung Extremalstellen und Wendepunkte, Kurvendiskussion}
\subsection{Regel von L’Hospital}
\subsection{Bestimmung von Nullstellen/Fixpunkten: direkte Fixpunktiteration, Newton-Verfahren}
\subsection{Taylorapproximation I: Taylor-Polynom n-ten Grades, Lagranges Restglied, Beispiele}
\subsection{Umkehrfunktionen und ihre Ableitung, Spezielle Funktionen: Wurzelfunktion, Umkehrfunktion der trigonometrischen Funktion, Exponential- und Logarithmusfunktion}
\subsection{Hyperbelfunktionen}

\section{Integration}
\subsection{Bestimmtes Integral: Unter- und Obersummen, Eigenschaften, Riemannsches Integral, Beispiele}
\subsection{Eigenschaften des Riemann-Integrals: Monotonie, Linearität}
\subsection{Riemannsches Kriterium zur Integrierbarkeit, beschränkte monotone/stetige Funktionen sind integrierbar}
\subsection{Mittelwertsatz der Integralrechnung}
\subsection{Anwendung: Volumenberechnung bei Rotationskörpern}
\subsection{Begriff der Stammfunktion, Hauptsatz der Differential- und Integralrechnung}
\subsection{unbestimmtes Integral, Beispiele}
\subsection{Partielle Integration, Substitutionsregel, Partialbruchzerlegung, Beispiele}
\subsection{Uneigentliche Integrale}

\section{Potenzreihen}
\subsection{Reihen, Beispiele, Rechenregeln, Konvergenz, Majorantenkriterium}
\subsection{Reihen von Funktionen:
Grenzfunktion
Konvergenzkriterien:
Stetigkeit
und
Cauchy,
Differentiation/Integration}
\subsection{Potenzreihen, Konvergenzradius und dessen Berechnung}
\subsection{Potenzreihendarstellung einiger Standardfunktionen, Binomialreihe}
\subsection{Taylorreihe einer Funktion, Anwendungen}
