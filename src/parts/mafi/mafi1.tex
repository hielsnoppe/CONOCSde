%
%
%
\section{Boolesche Aussagenlogik}

\subsection{Grundbegriffe; Vom Booleschen Term zur Booleschen Funktion}
\subsection{Von der Booleschen Funktion zum Booleschen Term}
\subsection{Der Gebrauch von Quantoren}
%
%
%
\section{Einführung Mengenlehre}
%
%
%
\section{Relationen und Funktionen}

\subsection{Grundbegriffe}
\subsection{Äquivalenzrelationen}
\subsection{Halbordnungsrelationen und totale Ordnungen}
\subsection{Funktionen}
\subsection{Abzählbarkeit}
%
%
%
\section{Mathematische Beweise; Vollständige Induktion}

\subsection{Das Schubfachprinzip von Dirichlet}
\subsection{Prinzipielles zu mathematischen Beweisen}
\subsection{Natürliche Zahlen und das Prinzip der vollständigen Induktion}
%
%
%
\section{Kombinatorik}

\subsection{Abzählen I}
\subsubsection{Binomialkoeffizienten}
\subsubsection{Binomialkoeffizienten und monotone Gitterwege}
\subsubsection{Mengenpartitionen}
\subsubsection{Zahlpartitionen}
\subsubsection{Doppeltes Abzählen}
\subsection{Die 12 Arten des Abzählens und ein Kartentrick}
\subsection{Diskrete Wahrscheinlichkeitsrechnung; Grundlagen}
\subsection{Erwartungswert, Spezielle Verteilungen}
\subsection{Das Coupon-Collector-Problem}
\subsection{Ein Random Walk und eine randomisierte Strategie}
\subsection{Abzählen III: Lineare Rekursionsgleichungen}
%
%
%
\section{Graphentheorie}

\subsection{Einführung und Grundlagen}
\subsubsection{Beispiele für algorithmische Aufgabenstellungen}
\subsubsection{Grundlegende Begriffe}
\subsection{Zusammenhang und Abstand in ungerichteten Graphen}
\subsection{Charakterisierung bipartiter Graphen}
\subsection{Bäume und ihre Charakterisierung}
\subsection{Grundlegende graphentheoretische Algorithmen}
\subsubsection{Graphdurchmustern: Breitensuche und Tiefensuche}
\subsubsection{Gerichtete azyklische Graphen}
\subsubsection{Einfache Anwendungen von Breiten- und Tiefsuche}
\subsection{Das Minimum-Spanning-Tree Problem: MST}
\subsubsection{Der MST-Algorithmus von Prim}
\subsubsection{Der MST-Algorithmus von Kruskal}
\subsection{Die Euler-Formel für planare Graphen; Maximales Matching}
\subsubsection{Die Euler-Formel}
\subsubsection{Reguläre Polyeder}
\subsubsection{Der Heiratssatz: Maximales Matching in bipartiten Graphen}
%
%
%
\section{Logik II}
\subsection{Der Resolutionskalkül}
\subsection{Hornformel und Einheitsresolventen}
\subsection{Algebraische Strukturen und Prädikatenlogik}
