%
%
%
\section{Einführung}

Dieser Kurs behandelt den Entwurf, die Analyse und Bewertung von Algorithmen.

Ein \textbf{Algorithmus} ist ein endlich beschriebenes effektives Verfahren,
das eine Eingabe in eine Ausgabe überführt.

Beispiele für Algorithmen sind etwa
Sortieralgorithmen (Quicksort, Mergesort, Heapsort, ...),
Suchalgorithmen (binäre Suche, Suche in speziellen Datenstrukturen) oder
Graphenalgorithmen (Kruskal, Floyd-Warshall, Bellman-Ford, ...).
Die meisten dieser genannten Algorithmen würde man jedoch nicht als höhere Algorithmen bezeichnen.
Ein Beispiel soll verdeutlichen, was einen sogenannten höheren Algorithmus ausmacht.

\subsection{Ein höherer Algorithmus}

Dazu betrachten wir ein Problem aus der Statistik.
Häufig kommt es vor, dass für eine große Datenmenge ein einzelner repräsentativer Wert gefunden werden soll.
Eine Möglichkeit dafür ist der Durchschnitt, der jedoch anfällig für Ausreißer ist.
Eine oftmals bessere Möglichkeit bietet der Median.

Der \textbf{Rang} $rg(s)$ eines Elementes $s \in S$
in einer total geordneten Menge $S$ ist die Anzahl von Elementen in $S$,
die kleiner sind, als $s$.

Der \textbf{Median} einer total geordneten Menge $S$ ist dasjenige Element $s \in S$,
dessen Rang $rg(s) = \ceil{\frac{\abs{S}+1}{2}}$ ist.
Das heißt, es gibt in $S$ genauso viele Elemente, die kleiner sind als $s$,
wie Elemente, die größer sind als $s$.

Eine einfache Möglichkeit den Median zu bestimmen ist,
die Menge zu sortieren und das Element an der Stelle $\ceil{\frac{\abs{S}+1}{2}}$ auszuwählen.
Dieser Ansatz hat eine Laufzeit von $\Theta(n \log n)$.

Ein weiterer Ansatz geht davon aus, dass es eine weitere Funktion $SPLITTER(S)$ gibt,
die zwar nicht genau den Median findet, jedoch eine gute Näherung in Form eines Elementes $q$,
für das gilt $\ceil{\frac{1}{4} n} \leq rg(q) \leq \floor{\frac{3}{4} n}$.

Wenn wir diese Funktion ohne weiteren Aufwand verwenden können, so lässt sich zeigen,
dass wir den Median in linearer Laufzeit bestimmen können.
Dazu lösen wir erst das allgemeinere Auswahlproblem.

\subsection{Das Auswahlproblem}

Finde ein Element $s \in S$ mit $rg(s) = k$.
Idee: Splitter zum Teilen verwenden, rekursiv vorgehen (ähnl. Quicksort)

\begin{algorithm}{SELECT}[S, k]{
}
\qif $\abs{S} < c$\\
	\qthen use brute force\\
	\qelse
		$q$ \qlet SPLITTER($S$)\\
		$S_<$ \qlet $\{s \in S | s < q\}$\\
		$S_>$ \qlet $\{s \in S | s > q\}$\\
		\qif $\abs{S_<} \geq k$\\
			\qthen \qreturn SELECT($S_<$, $k$)\\
			\qelse
				\qif $\abs{S_<} = k-1$\\
					\qthen \qreturn $q$\\
					\qelse \qreturn SELECT($S_>$, $k-\abs{S_<}-1$)
				\qfi
		\qfi
\qfi
\end{algorithm}

Laufzeitanalyse





\begin{algorithm}{SELECT}[S, k]{
}
\qif $\abs{S} < c$\\
	\qthen use brute force\\
	\qelse
		unterteile $S$ in 5er-Gruppen\\
		$S'$ \qlet Mediane aller 5er-Gruppen\\
		$q$ \qlet SELECT($S'$, $\ceil{\frac{\abs{S'}+1}{2}}$)\\
		$S_<$ \qlet $\{s \in S | s < q\}$\\
		$S_>$ \qlet $\{s \in S | s > q\}$\\
		\qif $\abs{S_<} > k$\\
			\qthen \qreturn SELECT($S_<$, $k$)\\
			\qelse
				\qif $\abs{S_<} = k$\\
					\qthen \qreturn $q$\\
					\qelse \qreturn SELECT($S_>$, $k-\abs{S_<}-1$)
				\qfi
		\qfi
\qfi
\end{algorithm}

\subsection{Church-Turing-These}

intuitiv berechenbar = RAM-berechenbar

\subsection{Laufzeit und Speicherplatz}

Gegeben ein RAM-Programm, das eine Funktion $f$ berechnet.
Sei $x \in \mathbb{Z}*$ eine Eingabe. Dann ist
die Laufzeit T(x) die Gesamtkosten der Arbeitsschritte bis das Programm bei Eingabe $x$ hält.
der Speicherplatz S(x) der gesamte Speicherbedarf

\subsubsection{Einheitskostenmaß}
\subsubsection{Logarithmisches Kostenmaß}

Pragmatische Entscheidung: EKM findet Anwendung bei kombinatorischen Algorithmen wie z.B. Sortieren, Suchen, Zeichenketten und Graphen.
LKM bei zahlentheoretischen Algorithmen wie z.B. Primzahltest.

Um allgemeine Aussagen über einen Algorithmus zu erhalten, fassen wir Eingaben $x$ nach ihrer Größe $n = \abs{x}$ zusammen.
Was genau dabei die Größe einer Eingabe darstellt, ist problemspezifisch festzulegen.
Für Zeichenketten bietet sich oft ihre Länge an, für Graphen etwa die Anzahl Knoten oder Kanten.
Um nicht alle verschiedenen Eingaben einer bestimmten Länge testen zu müssen, wird jeweils der worst-case angenommen,
also Laufzeit und Speicherbedarf im jeweils komplexesten Fall berechnet.
Dabei gilt für alle $x$ mit $\abs{x} = n$:
$$T_{worst-case}(n) = \max T(x) \qquad S_{worst-case}(n) = \max S(x)$$

\section{Grundlegende Techniken}

\subsection{Divide et impera}

\subsubsection{Lösen von Rekursionsgleichungen}

Es gibt verschiedene Ansätze um Rekursionsgleichungen zu lösen und in eine geschlossene Form zu bringen.

\paragraph{Intuition}
Mit viel Erfahrung oder Glück kann man das Ergebnis erraten und dessen
Richtigkeit anschließend z.B. mittels vollständiger Induktion beweisen.

\paragraph{Einsetzen} Wiederholt einsetzen und Muster erkennen

\paragraph{Rekursionsbaummethode} Ein Bild malen und Muster erkennen

\paragraph{Master-Theorem}
Dies ist ein allgemeines Rezept, das eine große Anzahl von Rekursionsgleichungen
abdeckt, jedoch nicht alle.

Sei $a \geq 1$, $b > 1$ und $f: \mathbb{N} \rightarrow \mathbb{N}$.\\
Sei $T(n) = a \cdot T(\frac{n}{b}) + f(n)$.

\subsubsection{Problem des engsten Punktpaares}

Gegeben sind $n$ Punkte in der Ebene $P \subseteq \mathbb{R}^2$, $\abs{P} = n$.
Gesucht sind diejenigen Punkte $p, q \in P, p \neq q$ für die $d(p, q)$ minimal ist.

\subsection{Dynamisches Programmieren}



\section{Datenstrukuren}

\subsection{Union-Find}

das Problem der Vereinigung disjunkter Mengen (DSU)

Gegeben ist eine endliche Menge $S = \{1, 2, \ldots, n\}$, sowie eine Partition
von $S$ in nichtleere disjunkte Teilmengen
$S = S_1 \cup S_2 \cup \ldots \cup S_k$

Gesucht ist eine Möglichkeit $S$ und die Partitionierung so zu speichern, dass
die Funktionen $UNION(S_i, S_j)$ und $FIND(s)$ effizient unterstützt werden:

$UNION(S_i, S_j)$: Ändere die Partitionierung, sodass $S_i$ und $S_j$ durch
$S_i \cup S_j$ ersetzt werden

$FIND(s)$: Finde die Menge aller Teilmengen von $S$, die $s \in S$ enthalten.

Anwendungen:
\begin{itemize}
\item Finden von Zusammenhangskomponenten in Graphen
\item Algorithmus von Kruskal
\item Segmentierung eines Bildes
\item Perkolation (?)
\item Äquivalenz in FORTRAN
\end{itemize}

Realisierung als Datenstruktur

Jedes Element entspricht einem Knoten/Record/Object
Jede Menge $S_i$ wird durch einen Repräsentanten $s \in S_i$ dargestellt.
Jedes Element $s$ hat einen Verweis auf einen nachfolger $s'$, der in der selben
Menge liegt, wie $s'$. Die Verweise führen zum Repräsentanten der jeweiligen
Menge, dessen Nachfolger $\perp$ ist.
Diese Darstellung heißt disjunkter Mengen-Wald. Jeder Baum entspricht einer
Menge der Partition. Jede Wurzel entspricht dem Repräsentanten der Menge.

Implementierung von UNION und FIND

FIND(s): Eingabe: Das Objekt, das das Element im Wald darstellt
Ausgabe: Wurzel des Baumes
Algorithmus: 


\subsection{Quake-Heaps}


\section{Hashing}

\subsection{universelles Hashing}
\subsection{perfektes Hashing}
\subsection{konsistentes Hashing}
\subsection{Anwendung Bloom-Filter}


\section{Graphen}

\subsection{Page-Rank}
\subsection{Max-Fluss-Min-Schnitt}


\section{Lineares Programmieren}

\subsection{Simplex}
\subsection{Dualität von Linearen Programmen}

\section{Komplexitätstheorie}
