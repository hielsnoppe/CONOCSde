%1. Datenmodellierung
%systematischer Entwurf von DB
%Schwerpunkt: Relationale DB
%2. Datenbanknutzung
%Zugriff auf die Daten mit SQL (Structured Query Language), interaktiv oder mit Anwendungsprogrammen.
%
%3. Implementierungsaspekte von DBS
%Transaktionen, Synchronisation, (Indexierung)
%
%4. Einführung in neuere Techniken der Datenverwaltung
%Data Warehouse (OLAP), Information Retrieval, Data Mining

%
%
%
\section{Einführung}

Eine \textbf{Datenbank} ist eine Menge von Datenobjekten,
die einem bestimmten \textit{Datenschema} genügen.

Ein \textbf{Datenbankschema} ist eine formale Beschreibung eines Teils der Wirklichkeit
mit den Begrifen eines bestimmten \textit{Datenmodells} (z.B. Tabellen und Klassen).

Ein \textbf{Datenmodell} ist eine Sprache um ein
\textit{Datenbankschema} (mittels der Data Definition Language) zu beschreiben,
sowie auf eine \textit{Datenbank} (mittels der Data Manipulation Language) zuzugreifen und sie zu bearbeiten.

%
%
%
\section{Thoretische Grundlagen}

\subsection{Schlüssel}

Ein \textbf{Superschlüssel} ist eine Menge von Attributen einer Relation,
für die keine zwei Tupel die gleichen Werte haben.
Jede Relation hat mindestens einen Superschlüssel.

% Was ist das?s
$t_1$, $t_2$ in the relation instance $r(R)$
$t_1$ [SK] $\neq$ $t_2$ [SK]

Ein \textbf{Schlüssel} $K$ einer Relation $R$ ist ein Superschlüssel von $R$,
von dem kein Attribut $\alpha$ entfernt werden kann,
ohne dass $K \setminus \alpha$ kein Superschlüssel von $R$ mehr ist.

Schlüssel sollen aus der Bedeutung der Attribute hergeleitet und nicht
aus den vorhandenen Werten der einzelnen Instanzen abgeleitet werden.
Eine Relation kann auch mehrere Schlüssel haben, die dann \textbf{Schlüsselkandidaten} heißen.
Von ihnen wird einer zum \textbf{Primärschlüssel} bestimmt, der dann verwendet wird,
um die Tupel der Relation eindeutig zu identifizieren.

Um in einer Relation Bezug auf einen Datensatz einer anderen Relation zu nehmen,
wird dessen Primärschlüssel $PK$ als sogenannter \textbf{Fremdschlüssel} $FK$ gespeichert.

Für einen Fremdschlüssel $FK$ in einer Relation $R_1$ muss gelten,
dass die Attribute in $FK$ die gleiche Bedeutung haben, wie in $PK$.

% Wie übersetzt man das sinnerhellend?
A value of FK in t1 in R1 :
either occurs as a value of PK for t2 in R2
i.e., t1 [FK] = t2 [PK]
or is null.


%
%
\subsection{Modellierung}

\subsubsection{Chen-Notation}

\paragraph{Entitäten}

(Rechteck),
schwache Entität (doppeltes Rechteck)

\paragraph{Relationen}

\paragraph{Attribute}

Attribute beschreiben Eigenschaften von Entitäten und Relationen.
Mn unterscheidet \textbf{einfache Attribute} (Oval, verbunden mit Entität),
\textbf{mehrwertige Attribute} (doppeltes Oval, verbunden mit Entität) und
\textbf{zusammengesetzte Attribute} (Oval, verbunden mit Attribut).
\textbf{Abgeleitete Attribute} (gestricheltes Oval).

\paragraph{}

\subsubsection{(min, max)-Notation}

%
%
\subsection{Funktionale Abhängigkeiten}

Funktionale Abhängigkeiten drücken Beziehungen zwischen Attributen einer Relation aus.
Eine Funktionale Abhängigkeit $F$ der Form $A \rightarrow B$ drückt dabei aus, dass das Attribut $A$ als Determinante von $F$ das Attribut $B$ eindeutig bestimmt.

\subsubsection{Armstrong-Axiome}

\begin{enumerate}
\item Reflexivität:\\ $X \subseteq Y \Rightarrow Y \rightarrow X$
\item Erweiterung / Verstärkung:\\ $X \rightarrow Y \Rightarrow XZ \rightarrow YZ$
\item Transitivität:\\ $X \rightarrow Y \land Y \rightarrow Z \Rightarrow X \rightarrow Z$
\item Vereinigung:\\ $X \rightarrow Y \land X \rightarrow Z \Rightarrow X \rightarrow YZ$
\item Dekomposition:\\ $X \rightarrow YZ \Rightarrow X \rightarrow Y \land X \rightarrow Z$
\item Pseudo-Transitivität:\\ $X \rightarrow Y \land YZ \rightarrow W \Rightarrow XZ \rightarrow W$
\end{enumerate}

\subsubsection{?}

Minimal cover of a set of FDs, Closure of attribute set

\subsubsection{kanonische Überdeckung}

Aus Wikipedia\footnote{\url{http://de.wikipedia.org/wiki/Kanonische_\%C3\%9Cberdeckung}}

Zu einer gegebenen Menge $F$ von funktionalen Abhängigkeiten nennt man $F_*$ eine kanonische Überdeckung, wenn folgende drei Eigenschaften erfüllt sind:
\begin{enumerate}
\item $F_* \equiv F$, das heißt $F_*^+ = F^+$
\item In $F_*$ existieren keine funktionalen Abhängigkeiten $\alpha \rightarrow \beta$, wobei $\alpha$ und $\beta$ Mengen überflüssiger Attribute enthalten; das heißt, es muss gelten:
\begin{enumerate}
\item $\forall A \in \mathbf \alpha: (F_* - (\alpha \rightarrow \beta) \cup ((\alpha - A) \rightarrow \beta)) \not\equiv F_*$
\item $\forall B \in \mathbf \beta: (F_*- (\alpha \rightarrow \beta) \cup (\alpha \rightarrow (\beta-B))) \not\equiv F_*$
\end{enumerate}
\item Jede linke Seite einer funktionalen Abhängigkeit in $F_*$ ist einzigartig. Dies kann durch fortgesetzte Anwendung der Vereinigungsregel auf funktionale Abhängigkeiten der Art $\alpha \rightarrow \beta$ und $\alpha \rightarrow \gamma$ erreicht werden, so dass die beiden funktionalen Abhängigkeiten durch $\alpha \rightarrow \beta\gamma$ ersetzt werden.
\end{enumerate}

\begin{enumerate}
\item Linksreduktion
\item Rechtsreduktion
\item Alle funktionalen Abhängigkeiten der Form $\alpha \rightarrow \{\}$ entfernen
\item Alle funktionalen Abhängigkeiten $\alpha \rightarrow \beta$ aus $F$ mit gleichem $\alpha$ zusammenfassen: Wenn $\alpha \rightarrow \beta \in F$, $\alpha \rightarrow \gamma \in F$, dann entferne diese beiden funktionalen Abhängigkeiten aus $F$ und füge $\alpha \rightarrow \beta\gamma$ zu $F$ hinzu.
\end{enumerate}

\subsubsection{Attributhülle}
Die Attributhülle $A^+$ eines Attributes $A$ bezeichnet die Menge all derjenigen Attribute, die von $A$ funktional abhängig sind.

\subsubsection{abgeschlossene Hülle}
Die abgeschlossene Hülle $F^+$ einer Menge von funktionalen Abhängigkeiten $F$ bezeichnet die Menge all derjenigen Attribute, die sich von den funktionalen Abhängigkeiten aus $F$ und deren Determinanten bestimmen lassen.

%
%
\subsection{Normalisierung}

\subsubsection{Erste Normalform (1. NF)}

Voraussetzung für die erste Normalform ist, dass alle Attribute atomar sind.
What we called relation so far.

Nicht in erster Normalform (N1NF) sind geschachtelte Relationen, in denen Attribute wiederum Relationen sind.

\subsubsection{Zweite Normalform (2. NF)}

Historical.

\subsubsection{Dritte Normalform (3. NF)}

Ein relationales Schema $R$ ist in 3. NF wenn, immer wenn eine funktionale Abhängigkeit X $\rightarrow$ An holds in R, entweder X ein Superschlüssel von R ist, oder A ein Primattribut von R ist.

Theorem:
Jede Relation in 1. NF besitzt eine ”well-behaved“ Dekomposition, i.e., in 3. NF, with lossless join that preserves dependencies.

Assume F is a minimal cover.
Algorithm to achieve a well-behaved 3NF decomposition.

1. For each FD (X $\rightarrow$ A) in F create a relation with schema (XA).

2. If none of the keys appears in one of the schemas of 1 then add a relation with schema Y, with Y a key.

3. If for relations created in 1. there exists a relation R1 whose schema is included in the schema of another relation, then remove R1.

4. Replace relations (X A1), ..., (X Ak) with a single relation (X A1 ... Ak).

\subsubsection{Boyce-Codd-Normalform (BCNF)}

A relation is in BCNF if for each $X \rightarrow A$ in F+, X is a superkey.

Theorem:
BCNF $\Rightarrow$ 3NF $\Rightarrow$ 2NF.

\subsubsection{Vierte Normalform (4. NF)}

Später, nicht Klausur-relevant.

%
%
%
\section{Sprachen zur Abfrage und Manipulation von Datenbanken}

%
%
\subsection{Relationale Algebra}

\subsubsection{Vereinigung $\cup$}

\subsubsection{Schnittmenge $\cap$}

\subsubsection{Differenz $\setminus$}

\subsubsection{Kartesisches Produkt $\times$}

\subsubsection{Projektion $\pi$}

\subsubsection{Selektion $\sigma$}

\subsubsection{Join $\bowtie$}

%
%
\subsection{Structured Query Language (SQL)}

Structured Query Language (SQL) ist eine formale Sprache, die Möglichkeiten zur Anfrage und Manipulation von Datenbanken auf verschiedenen Ebenen bietet.

Auf der strukturellen Ebene können innerhalb eines DBMS ganze Datenbanken und innerhalb einer Datenbank Tabellen erstellt, verändert oder gelöscht werden.

Auf der inhaltlichen Ebene ermöglicht SQL Datensätze zu lesen, bearbeiten und zu löschen.

\subsubsection{Strukturelle Operationen}

CREATE / CHANGE / DROP DATABASE
CREATE / CHANGE / DROP TABLE

\subsubsection{Inhaltliche Operationen}

SELECT
UPDATE
DELETE

Aggregatfunktionen
SUM, AVG, MAX, MIN, COUNT
DISTINCT
GROUP BY
UNION

GROUPING SETS
CUBE
ROLLUP

\subsubsection{Datentypen}

DATE, INT, VARCHAR, TEXT

%
%
%
\section{Technische Umsetzung}

%
%
\subsection{Verarbeitung und Optimierung von Anfragen}

Schritte der Verarbeitung

\begin{enumerate}
\item Query in a high-level query language\\
SCANNING, PARSING, VALIDATING
\item Intermediate form of a query\\
QUERY OPTIMIZER
\item Execution plan\\
QUERY CODE GENERATOR
\item Code to execute the query (interpreted or compiled)\\
RUNTIME DB PROCESSOR
\item Result of query

\end{enumerate}

\subsubsection{SCANNING, PARSING, VALIDATING}

\subsubsection{QUERY OPTIMIZER}

Query-Trees
Perform selection operation as early as possible.
Replace expressions of the form
$\sigma P1 \land P2 (e)$
with
$\sigma P1 (\sigma P2 (e))$
Perform projections early

\subsubsection{QUERY CODE GENERATOR}

\subsubsection{RUNTIME DB PROCESSOR}

%
%
\subsection{Transaktionen}

%
%
\subsection{Nebenläufigkeit}
