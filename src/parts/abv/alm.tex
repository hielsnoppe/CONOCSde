%
%
%
\section{Einführung}
%
%
%
\section{Ziele oder Wünsche, Handeln oder Nur-Überlegen, Lösungen oder Probleme}
%
%
%
\section{Subjektivität oder Objektivität, Freude oder Unaufmerksamkeit, Freiwilligkeit oder Zwang, Dankbarkeit oder Ärger}
%
%
%
\section{Vertrauen oder Angst}
%
%
%
\section{Konzentration oder Zeitverschwendung, Entspannung oder Stress}
%
%
%
\section{Entscheiden oder Zagen, Verantwortung oder Fremdbestimmung}
%
%
%
\section{Programme entdecken und bearbeiten}
%
%
%
\section{Klarheit oder Überraschung, Zuwendung oder Gleichgültigkeit}
%
%
%
\section{Motivation oder Unlust, Akzeptieren oder Sich-Abfinden}
%
%
%
\section{Werte oder Beliebigkeit, Konsequenz oder Scheitern}

\subsection{Begriffe}

\paragraph{Wert} Eine Idee, die mir dauerhaft wichtig ist.
Erkennbar daran, dass ich konsequent danach handele.

\paragraph{Konsequenz} besteht darin, Werte im Handeln nicht zu verletzen und nicht dauerhaft zu vernachlässigen.

\paragraph{Lebensaufgabe} Ein zeitlich und inhaltlich unbegrenzter Wunsch (also kein Ziel), dem ich mein Leben lang als Leitlinie für mein Verhalten folge.

\subsection{Werte}

Ablenkung
Anderssein
Anerkennung
Angstfreiheit
Anstand
Attraktivität
Ausgeglichenheit
Autarkie
Autonomie
Balance
Benehmen
Beschäftigung
Chancengleichheit
Chaos
Demokratie
Disziplin
Ehrgeiz
Einfluss
Eleganz
Engagement
Erfolg
Erwartung
Fairness
Familie
Fitness
Fleiß
Flexibilität
Freiheit
Freizeit
Freude
Führung
Fürsorge
Geborgenheit
Gelassenheit
gerechtigkeit
Gesundheit
Gleichberechtigung
Gleichheit
Glück
Härte
Heimat
Hilfsbereitschaft
Hoffnung
Individualität
Kinder
Kooperation
Kompromiss
Konfliktfähigkeit
Konsequenz
Kontrolle
Kraft
Kultur
Leben
Liebe
Macht
Nachhaltigkeit
Naturschutz
Nächstenliebe
Offenheit
Ordnung
Privatsphäre
Pünktlichkeit
Qualität
Reichtum
Respekt
Ruhe
Schmerz
Schönheit
Selbstbestimmung
Sicherheit
Sitte
Solidarität
Sorge
Spaß
Sport
Streben, Strebsamkeit, Zielstrebigkeit
Tierschutz
Toleranz
Tradition
Trauer
Überleben
Überlegenheit
Umweltschutz
Unauffälligkeit
Verantwortung
Vergnügen
Verlangen
Verlässlichkeit
Vernunft
Vertrauen
Vorbildsein
Vorfreude
Wahrheit
Wehmut
Wertschätzung
Wissen
Wohlfühlen
Zeitvertreib
Zuverlässigkeit
Zweckmäßigkeit

\subsection{Lebensaufgabe}
Ein Ansatz: Wenn ich Nichts mehr tun muss, aber alles tun darf und kann, was tue ich?
Dinge / Wünsche, die einen motivieren / ansprechen, aber nicht zeitlich befristet sind (also nicht zur Zielsetzung geeignet), sind Kandidaten für Werte oder eine Lebensaufgabe.

\subsection*{Gewohnheiten verschieben}
Ich üernehme immer mehr Verantwortung für mein eigenes Leben
Ich weiß, was ich will.
\subsection*{Hausaufgabe}
Ich finde die drei wichtigsten Beispiele für Konsequenz oder Mangel daran aus meinem Leben.
Was war jeweils die Folge daraus?
Ich formulliere eine Hypothese, was meine Lebensaufgabe sein könnte.

%
%
%
\section{Alles oder Nichts}

\subsection*{Hausaufgabe}
Ich ergänze oder ändere meine Zielsetzung so, dass mein notleidenster Lebensbereich deutlich mehr Geltung bekommt.
%
%
%
\section{Soft-Skills}

Zeitmanagement
Mediation, Konfliktmanagement
Moderation, Gesprächsführung
Teamarbeit
Rhetorik, Präsentation

\subsection{Multiple Intelligenzen}
\begin{itemize}
\item logisch-mathematische
\item sprachlich-linguistische
\item bildlich-räumliche
\item körperlich-kinästhetische
\item musikalisch-rhythmische
\item interpersonelle
\item intrapersonelle
\end{itemize}

\subsection{Kompetenzerwerb}
unbewusste Inkompetenz
bewusste Inkompetenz
bewusste Konpetenz
unbewusste Kompetenz

\subsection*{Gewohnheiten verschieben}
Ich lerne hinzu, indem ich in der Praxis übe und Könner frage bzw. ein Buch lese.
Ich pflege auf jedem Sektor einen Stil, der zu mir passt.
Ich werde vor allem meine Stärken stärken, anstatt gegen meine Schwächen anzugehen.
