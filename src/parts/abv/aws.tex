%
%
%
\section{Meta-Ebene}
\subsection{Benutzbarkeit}

Brauchbarkeit / Benutzbarkeit
	Erlernbarkeit
	Verständlichkeit
	Bedienbarkeit
		Übersichtlichkeit / Intuitivität
		Konsistenz
		Fehlerunanfälligkeit
		Bequemlichkeit / Geschwindigkeit
%
%
%
\section{Gesamtgesellschaftliche Wirkungen}
%
%
%
\section{Sicherheit}

\subsection{Fehlerbaumanalyse}

Vorgehen:
Fehler-Fall als oberstes Ereignis (TOP-Ereignis)
Verursachende Ereignisse bestimmen, dabei
	grundlegende Ereignisse (basic events) verwenden
	Einteilung von Ereignissen in korrigierbare und nicht-korrigierbare.
	Verknüpfung von Ereignissen durch logische Operatoren:
		AND, OR, XOR, Bedingte Verknüpfungen?
	paarweise stochastische Unabhängigkeit beachten

\begin{tabular}{cll}
AND	& Und-Gatter (and-gate)
	& alle Ursachen müssen vorliegen\\
OR	& Oder-Gatter (or-gate)
	& mindestens eine Ursache muss vorliegen\\
XOR	& Entweder-Oder-Gatter
	& Genau eine Ursache muss vorliegen\\
BLOCK	& Block-Gatter (inhibit-gate)
	& Nebenbedingung und die Ursache müssen vorliegen\\
INTEVT	& Zwischenereignis (intermediate event)
	& besteht aus einer Kombination von Unterereignissen\\
BSCEVT	& Basis-Ereignis (basic event)
	& wird nicht weiter untersucht\\
UDVEVT	& nicht untersuchtes-Ereignis (undeveloped event)
	& notwendige Informationen sind (noch) nicht vorhanden\\
TRANS	& Transfer-Symbol (transfer symbol)
	& verbindet Fehlerbäume
\end{tabular}
Quelle: http://opus.bibliothek.uni-augsburg.de/volltexte/2004/38/pdf/eVeroeffentlichung.pdf
%
%
%
\section{Privatsphäre}

\subsection{Definitionen und Beispiele}

\subsection{Gesetzliche Regelungen}
Wie in den Folien auch hier der Hinweis,
dass die Angaben stark vereinfacht sind und keinen Anspruch auf Vollständigkeit haben.
Die rechtsverbindliche Fassung des Gesetzes ist unbedingt dem jeweils aktuellen Gesetzestext zu entnehmen.
\subsubsection{Grundsätze}
\subsubsection{Datenschutzgesetz}
\paragraph{BDSG §1: Zweck des Gesetzes}
\paragraph{BDSG §3: Begriffsbestimmungen}
\paragraph{BDSG §3a: Datenvermeidung und Datensparsamkeit}
\paragraph{BDSG §4: Zulässigkeit der Erhebung und Nutzung}
\paragraph{BDSG §6: Unabdingbare Rechte des Betroffenen}
\paragraph{BDSG §13: Erhebung}
\paragraph{BDSG §14: Speicherung und Nutzung}
\paragraph{BDSG §19: Auskunft an den Betroffenen}
\paragraph{BDSG §20: Berichtigung und Löschung}
\paragraph{BDSG §27: Anwendungsbereich}
\paragraph{BDSG §28: Erhebung und Speicherung für eigene Zwecke}
\paragraph{BDSG §29: Geschäftsmäßige Datenerhebung und -speicherung zum Zweck der Übermittlung}
\paragraph{BDSG §30: Geschäftsmäßige Datenerhebung und -speicherung zur Übermittlung in anonymisierter Form}
\paragraph{BDSG §31: Besondere Zweckbindung}
\paragraph{BDSG §32: Beschäftigungsverhältnis}
\paragraph{BDSG §33, §42a: Benachrichtigung des Betroffenen}
\paragraph{BDSG §34: Auskunft an den Betroffenen}
\paragraph{BDSG §35: Berichtigung, Löschung und Sperrung von Daten}
\paragraph{BDSG §43, §44: Bußgeld- und Strafvorschriften}
%
%
%
\section{Arbeitswelt}
\subsection{Entschiedungsprozesse}
\paragraph{Entscheidungsprozess} Eine Abfolge von Ereignissen, die mit der Wahrnehmung eines Problemes und dem Entschluss dafür eine Lösung zu finden beginnt. Die einzelnen Ereignisse sind sowohl explizite Entscheidungen, die die beteiligten Personen treffen, als auch implizite Entschidungen, die aufgrund von Sachzwängen und dem situativen Kontext getroffen werden.
\paragraph{Mikropolitik}
\paragraph{(Macht-)spiele} Routine- und Innovationsspiele
\paragraph{Changemanagement} radikale, schnelle Veränderung (Reform) im Gegensatz zu Organisationsentwicklung im Sinne von Evolution
%
%
%
\section{Aus dem Tutorium}
\subsection{Struktogramme}
Erstellung eines Struktogrammes
\begin{enumerate}
\item Begriffe nach Grad der Spezialisierung ordnen
\item Mit dem allgemeinsten Begriff beginnen
\item Übrige Begriffe hinzufügen und Beziehungen zu anderen Begriffen kennzeichnen
\item Verwandte / ähnliche Begriffe sollten dabei in räumlicher Nähe zueinander stehen.
\end{enumerate}
