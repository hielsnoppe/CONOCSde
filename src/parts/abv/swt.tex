%
%
%
\section{Einführung}
Software; Softwaretechnik (SWT); Aufgaben der SWT; Beteiligte;
Gütemaßstab: Kosten/Nutzen; Qualität; Produkt und Prozess; Prinzip, Methode, Verfahren, Werkzeug;
technische vs. menschliche Aspekte; Arten von SWT-Situationen; Lernziele; Lernstil
"Merke"-Hinweise zu: Domänen, nichtfunktionalen Anforderungen, Kooperationsbedarf,
Projektrisiko.

Die Welt der Softwaretechnik
Routine und Innovation: Normales und radikales Vorgehen; Taxonomie: Probleme und Lösungen 

\section{Modellierung und UML}
Modelle und Modellierung (Realität vs. Modell; Phänomene vs. Konzepte); UML;

Klassendiagramme
Klassen, Attribute, Methoden
Vererbung, Aggregation, Komposition

Objektdiagramme

Sequenzdiagramme

Zustandsdiagramme (statechart)

Aktivitätsdiagramme

Anwendungsfalldiagramme

Komponentendiagramme

Kollaborationsdiagramme,
Inbetriebnahmediagramme,
Kommunikationsdiagramme,
Interaktions-Übersichts-Diagramme

UML Metamodell; Profile;
Notationsdetails
(Klassen, Assoziationen, Schnittstellen, Zustände) 

\section{Ermitteln WAS}

\subsection{Anforderungsbestimmung}
Erhebung (Requirements Elicitation): Anforderungen und Anforderungsbestimmung
(Requirements Engineering); Arten von Anforderungen; Anforderungen und Modellierung;
Harte und weiche Systeme; Probleme und Chancen erkennen;
Erhebungstechniken (herkömmliche, darstellungs-basierte, soziale, wissenserhebende) 

\subsubsection{Anwendungsfälle (Use Cases)}

Was ist ein Use Case?;
Wichtige Parameter (Bereich, Detailgrad/Zielniveau);
schrittweise Präzisierung;
Use-Case-Hierarchien (Überblick, Benutzerziele, Details);
SuD System under discussion

Checkliste für Use Cases
\begin{description}
\item[Titel]
\begin{itemize}
	\item Aktive Verbalphrase, die das Ziel des Hauptakteurs nennt?
	\item Ist das SuD für die Zielerreichung "zuständig"?
	\end{itemize}
\item[Bereich (SuD)]
\begin{itemize}
	\item Angegeben?
	\item Wenn das SuD entworfen werden muss, müssen (a) alle Teile entworfen werden und (b) nichts außerhalb (Systemgrenze)?
	\end{itemize}
\item[Detailgrad/Zielniveau]
\begin{itemize}
	\item Liegt das Ziel wirklich auf diesem Niveau?
	\item Passt der Inhalt zum geplanten Detailgrad?
	\end{itemize}
\item[Hauptakteur]
\begin{itemize}
	\item Hat er/sie/es Verhalten?
	\item Hat er/sie/es ein Ziel, das zu einer Dienstzusicherung des SuD passt?
	\end{itemize}
\item[Voraussetzungen]
\begin{itemize}
	\item Sind sie verbindlich und herstellbar?
	\item Werden sie im Use Case nicht mehr überprüft?
	\end{itemize}
\item[Beteiligte und ihre Interessen]
	Muss das SuD in diesem Use Case diese Interessen bedienen?
\item[Mindestzusicherungen]
	Sind die Interessen aller Beteiligten angemessen geschützt?
\item[Zusicherungen im Erfolgsfall]
	Sind die Interessen aller Beteiligten befriedigt?
\item[Haupt-Erfolgsszenario]
\begin{itemize}
	\item Hat es 3 bis 9 Schritte?
	\item Beschreibt es den Ablauf vom Auslöser bis zur Erfolgsgarantie?
	\item Erlaubt es ggf. geeignete Abwandlungen in der Reihenfolge?
	\end{itemize}
\item[Jeder Einzelschritt]
\begin{itemize}
	\item Ist er als erreichtes Ziel formuliert?
	\item z.B. "validieren", nicht: "prüfen"
	\item Treibt er den Prozess sichtbar voran?
	\item Ist klar, wer der handelnde Akteur ist?
	\item Ist die Absicht des Akteurs klar?
	\item Abstrahiert der Schritt von der Bedienschnittstelle?
	\item Ist erkennbar, welche Information verarbeitet wird?
	\end{itemize}
\item[Erweiterungsbedingungen]
\begin{itemize}
	\item Kann das SuD diesen Fall entdecken (falls nötig)?
	\item Muss das SuD diesen Fall behandeln?
	\end{itemize}
\item[Inhalt des Use Cases insgesamt]
\begin{itemize}
	\item Gegenüber den Beteiligten: "Ist dies, was Du möchtest?",
	"Kannst Du später entscheiden, ob es richtig gebaut wurde?"
	\item Gegenüber den Entwicklern: "Kannst Du das bauen?"
	\end{itemize}
\end{description}

Anwendungsfalldiagramme

\section{Verstehen WAS}

\subsection{Analyse (statisches Objektmodell)}
Von Use-Cases zu Klassen,

Abbott's Methode (Substantive sind Kandidaten für Klassen, Verben für Operationen,
Adjektive für Attribute, Eigennamen für Objekte, "ist ein" für Vererbung etc.);

Checklisten zur Identifikation von Klassen, Assoziationen, Attributen, Operationen, Vererbung;
Entwicklerrollen und Modellarten (Analysemodell vs. Entwurfsmodell) 

\subsection{Analyse (dynamisches Objektmodell)}
Klassen finden mit dynamischer Modellierung; Zustandsdiagramme (statechart diagrams);
Sequenzdiagramme; Aufbau eines Anforderungsanalyse-Dokuments;
Validierung (und Gegensatz zu Verifikation) 

\section{Entscheiden WIE}

\subsection{Software-Architektur}
Architektur=Gesamtstruktur; Erfüllen nichtfunktionaler und funktionaler Anforderungen;
globale Eigenschaften; wiederverwendbare Architekturen (Standard-Architekturen);
Architekturstile (zum Selbstentwickeln von Architekturen);
Modularisierung (Modulbegriff, Aufteilungskriterien)

Es gibt für eine Reihe wiederkehrender Mengen
nichtfunktionaler Anforderungen etablierte SW-Architekturen
oder zumindest Architekturstile
\begin{itemize}
\item Klient-/Dienstgeber-Architektur (client/server arch.)
\item eine einfache Sorte verteilter Architekturen
\item Ereignisgesteuertes System (event-based arch.)
\item eine Sorte lose gekoppelter Architekturen
\item Ablage-basierte Architektur (repository arch.)
\item noch eine Sorte lose gek. A.; oft mit Ereignissteuerung verbunden
\item Unterbrechungsorientiertes System (interrupt-based system)
\item eine Architektur für kleinere Echtzeitsysteme
\item Mehrschicht-Architektur (layered arch.)
\item ein allgemeiner A.stil, der mit vielen anderen A.ideen verbunden werden kann
\item JavaEE-Architektur, CORBA-Architektur, .NET-Architektur
\item technologiezentrierte Architekturen
\end{itemize}

\subsection{Modularisierung}
Modulbegriff; Kriterien für Aufteilung;
Fallstudie: KWIC;
KWIC 1: Datenflusskette; Einschätzen der Entwurfsqualität;
KWIC 2: Zentrale Steuerung;
KWIC 3: Datenabstraktion; Verhalten bei Änderungen; Verwandtschaft mit Architekturstilen 

\subsection{Entwurfsmuster}
Was macht ein Problem schwierig?; Einfachheit durch Wiedererkennen von Mustern;
Idee von Entwurfsmustern;
Kompositum-Muster (composite pattern);
Adapter-Muster (adapter pattern);
Brücken-Muster (bridge pattern);
Fassaden-Muster (facade pattern) 

Arten von Entwurfsmustern;
Stellvertreter-Muster (proxy pattern);
Kommando-Muster (command pattern);
Beobachter-Muster (observer pattern);
Strategie-Muster (strategy pattern);
Abstrakte-Fabrik-Muster (abstract factory pattern);
Erbauer-Muster (builder pattern) 

\subsection{Schnittstellenspezifikation}
Sichtbarkeiten (public, protected, private, package),
Spezifikation von Voraussetzungen (preconditions) und Wirkungen (postconditions) mit OCL
(context, pre, post, inv);
Abbildung von Assoziationen in Code

\subsection{Qualitätssicherung}

Analytische Qualitätssicherung
Defekttest;
Auswahl der Eingaben (Funktionstest, Strukturtest); Auswahl der Testgegenstände (bottom-up, top-down, opportunistisch);
Ermittlung des erwarteten Verhaltens (Referenzsystem, (Teil)Orakel); Wiederholung von Tests (Rückfalltesten, Testautomatisierung) 
Testautomatisierung (Werkzeuge, Strukturierung, JUnit);
Stoppkriterien für das Testen;
Defektortung;
Benutzbarkeitstest;
Lasttest;
Akzeptanztest;
manuelle statische Prüfung (Durchsicht; Inspektion; Perspektiven-basiertes Lesen);
automatische statische Prüfung (Modellprüfung; Quelltextanalyse) 

Konstruktive Qualitätssicherung (Qualitätsmgmt., Prozessmgmt.)
Projekt- vs. Prozessmgmt.;
Arten von Prozessmgmt.-Leitlinien;
CMM-SW/CMMI (5 Prozessreifestufen);
Total Quality Management (TQM) (Prinzip: Kundenzufriedenheit); ISO 9000 

\subsection{Prozessmodelle}

Bestandteile eines Prozessmodelles:

Rollen
\begin{itemize}
\item Welche gibt es und wie genau sind sie definiert ("Arbeitsbeschreibungen")?
\end{itemize}

Aktivitäten
\begin{itemize}
\item Welche sind vorgesehen? Wie genau sind sie definiert?
\item Entscheidungsspielraum über Einsatz oder Art des Einsatzes?
\item Gibt es Checklisten? Vorgaben über Werkzeuge, Methoden, Richtlinien?
\end{itemize}

Artefakte
\begin{itemize}
\item Welche sind vorgesehen? Wie genau sind sie definiert? Wie verbindlich ist das?
\item Gibt es Vorlagen (Schablonen)?
\end{itemize}

Steuerung der Aktivitäten
\begin{itemize}
\item Gibt es einen festen Ablaufplan? Oder weichere Kriterien für die Abfolge von Aktivitäten?
\item Sind Eintritts- und Austrittsbedingungen definiert?
\end{itemize}

Prozessmodell-Auswahlkriterien

\subsubsection{Wasserfallmodell}

Das Wasserfallmodell ist ein lineares (nichtiteratives) Vorgehensmodell in der Softwareentwicklung, bei dem der Softwareentwicklungsprozess in Phasen organisiert wird.
Dabei gehen die Phasenergebnisse wie bei einem Wasserfall immer als bindende Vorgaben für die nächsttiefere Phase ein.

Im Wasserfallmodell hat jede Phase vordefinierte Start- und Endpunkte mit eindeutig definierten Ergebnissen.
In Meilensteinsitzungen am jeweiligen Phasenende werden die Ergebnisdokumente verabschiedet.
Zu den wichtigsten Dokumenten zählen dabei das Lastenheft sowie das Pflichtenheft.

\begin{enumerate}
\item Planung
\item Anforderungsbestimmung
\item Architekturentwurf
\item Feinentwurf
\item Implementierung
\item Integration
\item Validierung
\item Inbetriebnahme
\end{enumerate}

\subsubsection{Prototypmodell}

\subsubsection{Evolutionäre Modelle}

\subsubsection{Spiralmodell}

\subsubsection{Flexiblere Planung (Agile Methoden)}

\subsubsection{Anpassbare Prozessmodelle}
Rational Unified Process (RUP)
V-Modell XT;
Erklärung "Agile Methode" (XP) 


\subsection{Projektmanagement}
Was und wofür?; Aufgabenfelder;
Schätzen (Schätzverfahren; Funktionspunktschätzung); Todesmarschprojekte 
Zeit- und Ressourcenplanung;
Microsoft Project;
Critical Path Method (CPM);
Finden von Aufgabenzerlegungen;
Risikomanagement; Risikolisten; DOs and DON'Ts 

Teams; Sportteam oder Chor?; Organisationsstrukturen; Rollen; Kommunikationsstrukturen;
psychologische Faktoren; Schätzen von Wahrscheinlichkeiten;
Motivation; Attribution; Haltungen; soziale Einflüsse 
Projektplan; Projektleitung;

nichtlineare Dynamik (Brook's Gesetz; Selbstverstärkung von Qualitätsmängeln;
Teufelskreis von Qualität und Zeitdruck);
Kommunikation (geplant/ungeplant, synchron/asynchron); Medien; Besprechungen 

Normales Vorgehen maximieren: Wiederverwendung
Arten der Wiederverwendung (Produkt/Prozess; Gegenstand; Ziel);
Risiken und Abwägung; Hindernisse; Produktivität; Wiederverwendung für normales Vorgehen;
Muster; Arten von Mustern;
Prinzipien (Abstraktion, Strukturierung, Hierarchisierung, Modularisierung, Lokalität, Konsistenz, Angemessenheit, Wiederverwendung, Notationen);
Analysemuster 
Benutzbarkeitsmuster; Prozessmuster; Mustersprachen; Anti-Muster; Werkzeuge 

Wissen weitergeben:
Dokumentation
Arten von Dokumentation; Qualitätseigenschaften (übersichtlich, präzise, korrekt, hilfreich);
positive und negative Beispiele; Prinzipien (Selbstdokumentation, Minimaldokumentation);
Begründungsmanagement (Fragen + Vorschläge + Kriterien + Argumente ergeben Entscheidungen) 
