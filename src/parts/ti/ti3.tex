\section{Notizen aus dem letzten Tutorium}
\begin{verbatim}
Kernel, monolithisch vs. micro, Linux ansehen
Interrupts, Systemaufrufe, Kontextwechsel
Prozess vs. Thread
PCB, Organisation von Prozessen durch das OS, Verwaltung durch Queues
Memory Management, Freispeicherverwaltung
drei verschiedene Adressierungen: absolute, relative, virtuelle Adressen
Fragmentierung
Caching
Scheduling (preemptive, non-preemptive)
Netzwerke Layer 1 3 4 7
\end{verbatim}
%
%
%
\section{Betriebssysteme}

\subsection{Schichtenmodell}

In welche sieben Ebenen kann man ein Rechnersystem einteilen?
\cite{kreissl}
\begin{enumerate}
\item Anwendungsebene (Anwendersoftware)
\item Assemblerebene (Beschreibung von Algorithmen, Link \& Bind)
\item Betriebssystem (Speichermanagment, Prozesskommunikation)
\item Instruction Set Architecture (ISA, Adressierungsarten)
\item Microarchitektur (Risc, Cisc, Branch Prediction..)
\item Logische Ebene (Register, Schieber, Latches..)
\item Transistorebene (Transistoren, MOS)
\end{enumerate}

Aufgabe eines Betriebssystems ist die Verwaltung von
Speicher,
Prozessen (Scheduling, Sicherheit) und
Hardwareressourcen (I/O).

\subsection{Speicherverwaltung}

\subsubsection{Aufteilung (Paging)}
Verschiedene Verfahren, den verfügbaren Speicher aufzuteilen:
Feste Speicherblöcke (Fixed Partitions), Problem: Interne Fragmentierung
Speicherblöcke variabler Größe (Dynamic Partitions), Problem: Externe Fragmentierung
Dynamische Algorithmen (First-Fit, Next-Fit)
Buddy System (?)
Paging

\subsubsection{Verwaltung (Swapping, Replacement-Policies)}
LRU, FIFO, Clock, (Optimal)

\subsection{Dateisystem}

\subsection{Prozessverwaltung}

\subsubsection{Scheduling}

\subsection{I/O-Verwaltung}

Programmed I/O, busy waiting\\
Interrupt-driven I/O\\
Direct Memory Access (DMA)

%
%
%
\section{Kommunikationssysteme}

\subsection{Schichtenmodelle (OSI und TP/IP)}

Der Aufbau von Computernetzwerken wird durch verschiedene Modelle beschrieben.
Eines davon ist das OSI-Modell (Open Systems Interconnection Reference Model),
welches eine sehr feine Unterschiedung verschiedener Aufgaben bei der Netzwerkkommunikation macht.
Ein anderes ist das TCP/IP-Referenzmodell,
welches im heutigen Aufbau des Internets umgesetzt ist und einige Schichten des OSI-Modells zusammenfasst:

\paragraph{Anwendungsschicht}
OSI: Anwendung (Application), Darstellung (Presentation), Sitzung (Session)
\paragraph{Transportschicht}
OSI: Transport
\paragraph{Internetschicht}
OSI: Vermittlung (Network)
\paragraph{Netzzugangsschicht}
OSI: Bitübertragung (Physical), Sicherungsschicht (Data Link)
Mit dieser Schicht beschäftigt sich das TCP/IP-Modell nicht näher.
Behandelt werden sollen Ethernet, WLAN und Bluetooth.

\subsection{Netzzugangsschicht}
Aufgabe dieser Schicht ist es, die Übertragung von Signalen zu ermöglichen.
Signale sind die Veränderung einer physikalischen Größe über einer bestimmten Zeit.
Sie stellen die Abstraktion von Daten dar.
Zu dieser Übertragung wird ein Medium benötigt, über das die Signale übertragen werden.
Dieses können zum Beispiel verschiedene Arten von Kabeln sein,
oder auch weniger anschauliche Medien wie der leere Raum bei drahtloser Übertragung.


\subsection{Internetschicht}

\subsubsection{IP-Header}


\subsection{Transportschicht}

\subsubsection{TCP- und UDP-Header}


\subsection{Anwendungsschicht}

\subsection{Eine Reise durch das Internet}

Was passiert, wenn ich eine Seite im Browser öffne, bzw. eine E-Mail versende?
